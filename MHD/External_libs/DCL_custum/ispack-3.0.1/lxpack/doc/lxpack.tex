%***********************************************************************
% ISPACK FORTRAN SUBROUTINE LIBRARY FOR SCIENTIFIC COMPUTING
% Copyright (C) 1998--2019 Keiichi Ishioka <ishioka@gfd-dennou.org>
%
% This library is free software; you can redistribute it and/or
% modify it under the terms of the GNU Lesser General Public
% License as published by the Free Software Foundation; either
% version 2.1 of the License, or (at your option) any later version.
%
% This library is distributed in the hope that it will be useful,
% but WITHOUT ANY WARRANTY; without even the implied warranty of
% MERCHANTABILITY or FITNESS FOR A PARTICULAR PURPOSE.  See the GNU
% Lesser General Public License for more details.
% 
% You should have received a copy of the GNU Lesser General Public
% License along with this library; if not, write to the Free Software
% Foundation, Inc., 51 Franklin Street, Fifth Floor, Boston, MA
% 02110-1301 USA.
%***********************************************************************
\documentclass[a4paper]{scrartcl}

\title{Manual of LXPACK}
\author{}
\date{}

\begin{document}

\maketitle

\section{Outline}

This is a package of subroutines to compute the associated
Legendre function transform.
This package is an upper-level package based on MXPACK.

Let $m$ and $n$ be the zonal and the total wavenumbers of
the associated Legendre function, respectively, and 
let the truncation wavenumbers for them be $M$ and $N$,
respectively. Here, $N\ge M$ is assumed (when $N=M$, it corresponds
to the triangular truncation).
Using this notation, the associated Legendre function transform
under these truncation wavenumbers is written as follows.
\begin{equation}
G^m(\varphi)\equiv\sum^N_{n=|m|}s^m_nP^m_n(\sin\varphi).
\end{equation}
Here, $\varphi$ is the latitude and
$P^m_n(\mu)$ is the associated Legendre function normalized to 2,
which is defined as:
\begin{equation}
P^m_n(\mu)\equiv\sqrt{(2n+1)\frac{(n-|m|)!}{(n+|m|)!}}
\frac1{2^nn!}(1-\mu^2)^{|m|/2}
\frac{d^{n+|m|}}{d\mu^{n+|m|}}(\mu^2-1)^n,
\end{equation}
\begin{equation}
\int^1_{-1}\{P^m_n(\mu)\}^2d\mu=2.
\end{equation}

The forward associated Legendre function transform
is written as follows.
\begin{equation}
s^m_n=\frac12\int^{\pi/2}_{-\pi/2}G^m(\varphi)P^m_n(\sin\varphi)\cos\varphi
d\varphi.
\end{equation}

In numerical calculation, the forward associated Legendre function 
transform is approximated as follows by using the Gauss-Legendre
integration formula,
\begin{equation}
s^m_n=\frac12\sum^J_{j=1}w_jG^m(\varphi_j)P^m_n(\sin\varphi_j).
\end{equation}
Here, $\varphi_j$s are called Gaussian latitude, which is defined as
the $J$ zero points (sorted ascending order) of the Legendre
polynomial $P_J(\sin\varphi)$ (this corresponds to 
the associated Legendre function of $m=0$ without the normalization
factor (the factor in $\sqrt{\quad}$), and $w_j$ is 
the Gaussian weight corresponding to each Gaussian latitude, 
which is defined as,
\begin{equation}
w_j\equiv\frac{2(1-\mu_j^2)}{\{JP_{J-1}(\mu_j)\}^2}.
\end{equation}
Here, $\mu_j\equiv\sin\varphi_j$.
In a certain condition, this integration formula gives the 
perfect approximation or the same value as the original integration.

This subroutine package contains
subroutines that do backward transform of spectral data ($s^m_n$) 
to wave data for each latitude ($G^m(\varphi_j)$),
subroutines that do forward transform of
wave data for each latitude ($G^m(\varphi_j)$)
to spectral data ($s^m_n$), and ancillary subroutines.

Here, $\varphi_j$s are $J$ Gaussian latutudes defined above.
In the following explanations for subroutines, the notations below
are used.
\begin{center}
\begin{tabular}{ll}
\texttt{NN}:& $N$(the truncation wavenumber for $n$)\\
\texttt{NM}:& maximum value of $N$ to be used\\
\texttt{JM}:& $J$(the number of Gaussioan latitudes)\\
\texttt{N}:& $n$(the total wavenumber)\\
\texttt{M}:& $m$(the zonal wavenumber)\\
\texttt{J}:& $j$(the index for each Gaussian latitude)
\end{tabular}
\end{center}


%%%%%%%%%%%%%%%%%%%%%%%%%%%%%%%%%%%%%%%%%%%%%%%%%%%%%%%%%%%%%%%%%%%%%%

\section{List of subroutines}

\vspace{1em}
\begin{tabular}{ll}
\texttt{LXINIG} & Initialzation of array \texttt{PZ}.\\
\texttt{LXINIR} & Initialzation of array \texttt{RM}.\\
\texttt{LXINIW} & Initialzation of arrays \texttt{PM, JC}.\\
\texttt{LXTSWG} & Transform from spectral data to wave data (wave componet).\\
\texttt{LXTGWS} & Transform from wave data to spectral data (wave componet).\\
\texttt{LXTSZG} & Transform from spectral data to wave data (zonal componet).\\
\texttt{LXTGZS} & Transform from wave data to spectral data (zonal componet).\\
\texttt{LXTSWV} & Paired transform from two spectral data to two wave data (wave componet).\\
\texttt{LXTVWS} & Paired transform from two wave data to two spectral data (wave componet).\\
\texttt{LXTSZV} & Paired transform from two spectral data to two wave data (zonal componet).\\
\texttt{LXTVZS} & Paired transform from two wave data to two spectral data (zonal componet).\\
\texttt{LXINIC} & Initialzation of array \texttt{CM}.\\
\texttt{LXCSWY} & Conversion of spectral data for backward transform corresponding to \\
& latitudinal derivative (wave component).\\
\texttt{LXCYWS} & Conversion of spectral data for forward transform corresponding to \\
& latitudinal derivative (wave component).\\
\texttt{LXCSZY} & Conversion of spectral data for backward transform corresponding to \\
& latitudinal derivative (zonal component).\\
\texttt{LXCYZS} & Conversion of spectral data for forward transform corresponding to \\
& latitudinal derivative (zonal component).\\
\texttt{LXTSZP} & Transform from zonal spectral data to the grid
values at the poles.
\end{tabular}

%%%%%%%%%%%%%%%%%%%%%%%%%%%%%%%%%%%%%%%%%%%%%%%%%%%%%%%%%%%%%%%%%%%%%%
\section{Usage of each subroutine}

\subsection{LXINIG}

\begin{enumerate}

\item Purpose

An initialization routine for \texttt{LXPACK}.  
It initializes the array, \texttt{PZ},
which is used other subroutines in \texttt{LXPACK}

\item Definition

\item Synopsis 
    
\texttt{LXINIG(JM,PZ,IG)}
  
\item Parameters 

\begin{verbatim}
INTEGER(8) :: JM, IG
REAL(8) :: PZ(JM/2,5)
\end{verbatim}
      
\begin{tabular}{ll}
\texttt{JM} & Input. The number of Gaussian latitudes.\\
\texttt{PZ}  & Output. An array which is used in other routines in LXPACK.\\
\texttt{IG} & Input. Specify the kind of grids(See Remark).
\end{tabular}

\item Remark

(a) \texttt{JM} must be an even number ($\ge 2$).
Maximum efficiency is achieved by setting \texttt{JM} to be 
a multiple of 8 when you do make ISPACK with setting either
SSE=avx or SSE=fma.
When setting SSE=avx512, maximum efficiency is achieved
by setting 
\texttt{JM} to be a multiple of 16.

(b) If \texttt{IG=1}, the latitudinal nodes $(\varphi_j)$
and the corresponding weights $(w_j)$ are set to be the
Gaussian latitudes and the Gaussian weights.
If \texttt{IG=2}, the latitudinal nodes are set to be the
\texttt{JM} equispaced nodes that divide the total latitudinal
interval into \texttt{JM+1} subintervals equally. Here, 
the both poles are not included in the nodes.
The corresponding weights are set to the weights for
Clenshaw-Curtis quadrature.
If \texttt{IG=3}, the latitudinal nodes are set to be the
\texttt{JM-1} equispaced nodes that divide the total latitudinal
interval into \texttt{JM} subintervals equally. Here, 
the both poles are not included in the nodes and
the node at the equator is used doubly in the transform routines.
The corresponding weights are set to the weights for
Clenshaw-Curtis quadrature. Here, the weight for the node
at the equator is set to a half of the Clenshaw-Curtis quadrature
weight since
the node at the equator is used doubly in the transform routines.

Assuming that the truncation wavenumber of spectral data
used in transform routines is \texttt{NM},
for the spectral data to be completely restored (within a rounding
error) after backward and forward transforms,
\texttt{JM}$\ge$ \texttt{NM+1} must be satisfied when
\texttt{IG=1}, and
\texttt{JM}$\ge$ \texttt{2*NM+1} must be satisfied 
when \texttt{IG=2},
and
\texttt{JM}$\ge$ \texttt{2*(NM+1)} must be satisfied 
when \texttt{IG=3}.

(c) In \texttt{PZ}, 
   \texttt{PZ(J,1)}:  $\sin(\varphi_{J/2+j})$,
   \texttt{PZ(J,2)}:  $\frac12 w_{J/2+j}$, 
   \texttt{PZ(J,3)}:  $\cos(\varphi_{J/2+j})$,
   \texttt{PZ(J,4)}:  $1/\cos(\varphi_{J/2+j})$,
   \texttt{PZ(J,5)}:  $\sin^2(\varphi_{J/2+j})$,   
are contained.

\end{enumerate}

%---------------------------------------------------------------------
\subsection{LXINIR}

\begin{enumerate}

\item Purpose

An initialization routine for \texttt{LXPACK}.  
It initializes the array, \texttt{RM},
which is used other subroutines in \texttt{LXPACK}

\item Definition

\item Synopsis 
    
\texttt{LXINIR(NM,M,RM)}
  
\item Parameters 

\begin{verbatim}
INTEGER(8) :: NM, M
REAL(8) :: RM((NM-M)/2*3+NM-M+1)
\end{verbatim}
      
\begin{tabular}{ll}
\texttt{NM} & Input. Maximum value of $N$ (the truncation 
  number for $n$).\\
\texttt{M} & Input. zonal wavenumber $m$ 
(\texttt{M}$\ge$ 0 must hold).\\
\texttt{RM}  & Output. An array which is used in other routines in LXPACK.\\
\end{tabular}

\item Remark

\end{enumerate}

%---------------------------------------------------------------------
\subsection{LXINIW}

\begin{enumerate}

\item Purpose  

An initialization routine for \texttt{LXPACK}.  
It initializes the array, \texttt{PM, JC},
which is used other subroutines in \texttt{LXPACK}
  
\item Definition

\item Synopsis 
    
\texttt{LXINIW(NM,JM,M,PZ,PM,RM,JC)}

\item Parameters 

\begin{verbatim}
INTEGER(8) :: NM, JM, M, JC((NM-M)/8+1)
REAL(8) :: PZ(JM/2,5), PM(JM/2,2)
REAL(8) :: RM((NM-M)/2*3+NM-M+1)
\end{verbatim}
    
\begin{tabular}{lll}
\texttt{NM} & Input. Maximum value of $N$ (the truncation 
number for $n$).\\
\texttt{JM} & Input. The number of Gaussian latitudes.\\
\texttt{M} & Input. zonal wavenumber $m$ 
(\texttt{M}$\ge$ 0 must hold).\\
\texttt{PZ}  & Input. The array initialized by \texttt{LXINIG}.\\
\texttt{PM}  & Output. An array which is used in other routines in LXPACK.\\
\texttt{RM}  & Input. The array initialized by \texttt{LXINIR} with
specifying \texttt{M}.\\
\texttt{JC}  & Output. An array which is used in other routines in LXPACK.
\end{tabular}

\item Remark

  (a)
\texttt{JM} must be an even number ($\ge 2$).
Maximum efficiency is achieved by setting \texttt{JM} to be 
a multiple of 8 when you do make ISPACK with setting either
SSE=avx or SSE=fma.
When setting SSE=avx512, maximum efficiency is achieved
by setting 
\texttt{JM} to be a multiple of 16.

\end{enumerate}

%---------------------------------------------------------------------

\subsection{LXTSWG}

\begin{enumerate}

\item Purpose 

Transform from spectral data to wave data (wave componet).
  
\item Definition

For wave component, transform spectral data ($s^m_n$) 
to wave data ($G^m(\varphi_j)$) at each latitude
by the backward associated Legendre function transform (see Outline).

\item Synopsis 

\texttt{LXTSWG(NM,NN,JM,M,S,G,PZ,PM,RM,JC,IPOW)}
  
\item Parameters

\begin{verbatim}
INTEGER(8) :: NM, NN, JM, M, JC((NM-M)/8+1), IPOW
REAL(8) ::  S(2,M:NN), G(2,JM), PZ(JM/2,5), PM(JM/2,2), RM((NM-M)/2*3+NM-M+1)
\end{verbatim}
  
\begin{tabular}{ll}
\texttt{NM} & Input. Maximum value of $N$ to be used.\\
\texttt{NN} & Input. $N$.\\
\texttt{JM} & Input. $J$.\\
\texttt{M} & Input. zonal wavenumber $m$ 
(\texttt{M}$\ge$ 0 must hold).\\
\texttt{S} & Input. Array that contains $s^m_n$.\\
\texttt{G} & Output. Array to contain $G^m(\varphi_j)$.\\
\texttt{PZ}  & Input. Array initialized by \texttt{LXINIG}.\\
\texttt{PM}  & Input. Array initialized by \texttt{LXINIW}
with specifying \texttt{M}.\\
\texttt{RM} & Input. Array initialized by \texttt{LXINIR}
with specifying \texttt{M}.\\
\texttt{JC} & Input. Array initialized by \texttt{LXINIW}
with specifying \texttt{M}.\\
\texttt{IPOW} & Input. 
The degree of $1/\cos\varphi$ multiplied 
simultaneously.\\
\end{tabular}

\item Remark

(a) \texttt{S(1,N)} and \texttt{S(2,N)} should 
contain $\mbox{Re}(s^m_n)$ and $\mbox{Im}(s^m_n)$, respectively.
$\mbox{Re}(G^m(\varphi_j))$ and 
$\mbox{Im}(G^m(\varphi_j))$ 
are to be contained in \texttt{G(1,J)} and \texttt{G(2,J)}, respectively.
  
(b) The array \texttt{G} must be aligned with 32byte boundary
if you did make ISPACK with setting either SSE=avx or SSE=fma
on Intel x86 CPUs and setting \texttt{JM} to be a multiple of 8.
The array \texttt{G} must be aligned with 64byte boundary
if you did make ISPACK with setting SSE=avx512
on Intel x86 CPUs and setting \texttt{JM} to be a multiple of 16.

(c) If you set \texttt{IPOW}$=l$, 
$(\cos\varphi_j)^{-l}G^m(\varphi_j)$ is returned 
in the output array \texttt{G} instead of $G^m(\varphi_j)$.

\end{enumerate}


%---------------------------------------------------------------------

\subsection{LXTGWS}

\begin{enumerate}

\item Purpose 

Transform from wave data to spectral data (wave componet).

\item Definition

For wave component, transform wave data ($G^m(\varphi_j)$) 
at each latitude
to spectral data ($s^m_n$) 
by the forward associated Legendre function transform (see Outline).

\item Synopsis 

\texttt{LXTGWS(NM,NN,JM,M,S,G,PZ,PM,RM,JC,IPOW)}

\item Parameters

(Since most of parameters are the same as in LXTSWG,
the following explanation is only for parameters
different from those in LXTSWG,)

\begin{tabular}{ll}
\texttt{S} & Output. Array to contain $s^m_n$.\\
\texttt{G} & Input. Array that 
contains $G^m(\varphi_j)$.\\
\texttt{IPOW} & Input. 
The degree of $1/\cos\varphi$ multiplied 
simultaneously with the transform. \\
&  An integer between 0 and 2.
\end{tabular}

\item Remark

  (a)
\texttt{G(1,J)} and \texttt{G(2,J)}should 
contain $\mbox{Re}(G^m(\varphi_j))$ and 
$\mbox{Im}(G^m(\varphi_j))$, respectively.
$\mbox{Re}(s^m_n)$ and $\mbox{Im}(s^m_n)$
are to be contained in \texttt{S(1,N)} and \texttt{S(2,N)}, respectively.

(b) If you set \texttt{IPOW}$=l$, 
$(\cos\varphi_j)^{-l}G^m(\varphi_j)$ is used as the input
instead of $G^m(\varphi_j)$.

\end{enumerate}

%---------------------------------------------------------------------

\subsection{LXTSZG}

\begin{enumerate}

\item Purpose 

Transform from spectral data to wave data (zonal componet).

\item Definition

For the zonal component, transform spectral data ($s^0_n$) 
to wave data ($G^0(\varphi_j)$) at each latitude
by the backward associated Legendre function transform (see Outline).

\item Synopsis 

\texttt{LXTSZG(NM,NN,JM,S,G,PZ,RM,IPOW)}
  
\item Parameters

\begin{verbatim}  
INTEGER(8) :: NM, NN, JM, IPOW
REAL(8) ::  S(0:NN), G(JM), PZ(JM/2,5), RM(NM/2*3+NM+1)
\end{verbatim}  

\begin{tabular}{ll}
\texttt{NM} & Input. Maximum value of $N$ to be used.\\
\texttt{NN} & Input. $N$.\\
\texttt{JM} & Input. $J$.\\
\texttt{S} & Input. Array that contains $s^0_n$.\\
\texttt{G} & Output. Array to contain $G^0(\varphi_j)$.\\
\texttt{PZ}  & Input. Array initialized by \texttt{LXINIG}.\\
\texttt{RM} & Input. Array initialized by \texttt{LXINIR}
with specifying \texttt{M=0}.\\
\texttt{IPOW} & Input. 
The degree of $1/\cos\varphi$ multiplied 
simultaneously.
\end{tabular}

\item Remark

(a) \texttt{S(N)} should 
contain $s^0_n$. $G^0(\varphi_j)$
is to be contained in \texttt{G(J)}.

(b) The array \texttt{G} must be aligned with 32byte boundary
if you did make ISPACK with setting either SSE=avx or SSE=fma
on Intel x86 CPUs and setting \texttt{JM} to be a multiple of 8.
The array \texttt{G} must be aligned with 64byte boundary
if you did make ISPACK with setting SSE=avx512
on Intel x86 CPUs and setting \texttt{JM} to be a multiple of 16.

(c) If you set \texttt{IPOW}$=l$, 
$(\cos\varphi_j)^{-l}G^0(\varphi_j)$ is returned 
in the output array \texttt{G} instead of $G^0(\varphi_j)$.

\end{enumerate}

%---------------------------------------------------------------------

\subsection{LXTGZS}

\begin{enumerate}

\item Purpose 

Transform from wave data to spectral data (zonal componet).

\item Definition

For the zonal component, transform wave data ($G^0(\varphi_j)$) 
at each latitude
to spectral data ($s^0_n$) 
by the forward associated Legendre function transform (see Outline).

\item Synopsis 

\texttt{LXTGZS(NM,NN,JM,S,G,PZ,RM,IPOW)}
  
\item Parameters

(Since most of parameters are the same as in LXTSZG,
the following explanation is only for parameters
different from those in LVTSZG,)

\begin{tabular}{ll}
\texttt{S} & Output Array to contain $s^0_n$.\\
\texttt{G} & Input. Array that contains $G^0(\varphi_j)$.\\
\texttt{IPOW} & Input. 
The degree of $1/\cos\varphi$ multiplied 
simultaneously.
\end{tabular}

\item Remark

(a) \texttt{G(J)} should 
contain $G^0(\varphi_j)$.
$s^0_n$ is to 
be contained in \texttt{S(N)}.

(b) If you set \texttt{IPOW}$=l$, 
$(\cos\varphi_j)^{-l}G^0(\varphi_j)$ is used as the input
instead of $G^0(\varphi_j)$.

\end{enumerate}

%---------------------------------------------------------------------

\subsection{LXTSWV}

\begin{enumerate}

\item Purpose

Paired transform from two spectral data to two wave data (wave componet).

\item Definition

For wave component, transform two spectral data ($s^m_n$)$_{(1,2)}$
to two corresponding wave data ($G^m(\varphi_j)$)$_{(1,2)}$ at each latitude
by the backward associated Legendre function transform (see Outline).

\item Synopsis 

\texttt{LXTSWV(NM,NN,JM,M,S1,S2,G1,G2,PZ,PM,RM,JC,IPOW)}
  
\item Parameters

\begin{verbatim}  
INTEGER(8) :: NM, NN, JM, M, JC((NM-M)/8+1), IPOW
REAL(8) ::  S1(2,M:NN), S2(2,M:NN), G1(2,JM), G2(2,JM)
REAL(8) ::  PZ(JM/2,5), PM(JM/2,2), RM((NM-M)/2*3+NM-M+1)
\end{verbatim}  

\begin{tabular}{ll}
\texttt{NM} & Input. Maximum value of $N$ to be used.\\
\texttt{NN} & Input. $N$.\\
\texttt{JM} & Input. $J$.\\
\texttt{M} & Input. zonal wavenumber $m$ 
(\texttt{M}$\ge$ 0 must hold).\\
\texttt{S1} & Input. Array that contains $(s^m_n)_1$.\\
\texttt{S2} & Input. Array that contains $(s^m_n)_2$.\\
\texttt{G1} & Output. Array to contain $(G^m(\varphi_j))_1$.\\
\texttt{G2} & Output. Array to contain $(G^m(\varphi_j))_2$.\\
\texttt{PM}  & Input. Array initialized by \texttt{LXINIW}
with specifying \texttt{M}.\\
\texttt{RM} & Input. Array initialized by \texttt{LXINIR}
with specifying \texttt{M}.\\
\texttt{JC} & Input. Array initialized by \texttt{LXINIW}
with specifying \texttt{M}.\\
\texttt{IPOW} & Input. 
The degree of $1/\cos\varphi$ multiplied 
simultaneously.\\
\end{tabular}

\item Remark

(a) \texttt{S1(1,N)} and \texttt{S1(2,N)} should 
contain $\mbox{Re}(s^m_n)_1$ and $\mbox{Im}(s^m_n)_1$, respectively.
Same as for \texttt{S2}.
$\mbox{Re}(G^m(\varphi_j))_1$ and 
$\mbox{Im}(G^m(\varphi_j))_1$ 
are to be contained in \texttt{G1(1,J)} and \texttt{G1(2,J)}, respectively.
Same as for \texttt{G2}.
  
(b) \texttt{G1, G2}
must be aligned with 32byte boundary
if you did make ISPACK with setting either SSE=avx or SSE=fma
on Intel x86 CPUs and setting \texttt{JM} to be a multiple of 8.
The arrays \texttt{G1, G2} must be aligned with 64byte boundary
if you did make ISPACK with setting SSE=avx512
on Intel x86 CPUs and setting \texttt{JM} to be a multiple of 16.

(c) If you set \texttt{IPOW}$=l$, 
$(\cos\varphi_j)^{-l}G^m(\varphi_j)_{(1,2)}$ are returned 
in the output arrays \texttt{G1, G2} instead of $G^m(\varphi_j)_{(1,2)}$.

\end{enumerate}

%---------------------------------------------------------------------

\subsection{LXTVWS}

\begin{enumerate}  

\item Purpose

Paired transform from two spectral data to two wave data (wave componet).

\item Definition

For wave component, transform two wave data ($G^m(\varphi_j)$)$_{(1,2)}$
to two corresponding spectral data ($s^m_n$)$_{(1,2)}$ at each latitude
by the backward associated Legendre function transform (see Outline).

\item Synopsis 
  

\texttt{LXTVWS(NM,NN,JM,M,S1,S2,G1,G2,PZ,PM,RM,JC,IPOW)}
  
\item Parameters

(Since most of parameters are the same as in LXTSWV,
the following explanation is only for parameters
different from those in LXTSWV)
  
\begin{tabular}{ll}
\texttt{S1} & Output. Array to contain $(s^m_n)_1$.\\
\texttt{S2} & Outpot. Array to contain $(s^m_n)_2$.\\
\texttt{G1} & Input. Array that contains $(G^m(\varphi_j))_1$.\\
\texttt{G2} & Input. Array that contains $(G^m(\varphi_j))_2$.\\
\texttt{IPOW} & Input. 
The degree of $1/\cos\varphi$ multiplied 
simultaneously.\\
\end{tabular}

\item Remark

  (a) \texttt{S1(1,N)} and \texttt{S1(2,N)}
are to contain
$\mbox{Re}(s^m_n)_1$ and $\mbox{Im}(s^m_n)_1$, respectively.
Same as for \texttt{S2}.
$\mbox{Re}(G^m(\varphi_j))_1$ and 
$\mbox{Im}(G^m(\varphi_j))_1$ should 
contain
\texttt{G1(1,J)} and \texttt{G1(2,J)}, respectively.
Same as for \texttt{G2}.

(b) \texttt{G1, G2}
must be aligned with 32byte boundary
if you did make ISPACK with setting either SSE=avx or SSE=fma
on Intel x86 CPUs and setting \texttt{JM} to be a multiple of 8.
The arrays \texttt{G1, G2} must be aligned with 64byte boundary
if you did make ISPACK with setting SSE=avx512
on Intel x86 CPUs and setting \texttt{JM} to be a multiple of 16.

(c) If you set \texttt{IPOW}$=l$, 
$(\cos\varphi_j)^{-l}G^m(\varphi_j)_{(1,2)}$ are used as the input
instead of $G^m(\varphi_j)_{(1,2)}$.

\end{enumerate}

%---------------------------------------------------------------------

\subsection{LXTSZV}

\begin{enumerate}

\item Purpose

Paired transform from two spectral data to two wave data (zonal componet).

\item Definition

For the zonal component, transform two spectral data ($s^0_n$)$_{(1,2)}$
to two corresponding wave data ($G^0(\varphi_j)$)$_{(1,2)}$ at each latitude
by the backward associated Legendre function transform (see Outline).

\item Synopsis 

\texttt{LXTSZV(NM,NN,JM,S1,S2,G1,G2,PZ,RM,IPOW)}

\item Parameters  
  
\begin{verbatim}  
INTEGER(8) :: NM, NN, JM, IPOW
REAL(8) ::  S1(0:NN), S2(0:NN), G1(JM), G2(JM)
REAL(8) ::  PZ(JM/2,5), RM(NM/2*3+NM+1)
\end{verbatim}

\begin{tabular}{ll}
\texttt{NM} & Input. Maximum value of $N$ to be used.\\
\texttt{NN} & Input. $N$.\\
\texttt{JM} & Input. $J$.\\
\texttt{S1} & Input. Array that contains $(s^0_n)_1$.\\
\texttt{S2} & Input. Array that contains $(s^0_n)_2$.\\
\texttt{G1} & Output. Array to contain $(G^0(\varphi_j))_1$.\\
\texttt{G2} & Output. Array to contain $(G^0(\varphi_j))_2$.\\
\texttt{PZ}  & Input. Array initialized by \texttt{LXINIG}.\\
\texttt{RM} & Input. Array initialized by \texttt{LXINIR}
with specifying \texttt{M=0}.\\
\texttt{IPOW} & Input. 
The degree of $1/\cos\varphi$ multiplied 
simultaneously.\\
\end{tabular}

\item Remark

(a) \texttt{S1(N)} and \texttt{S2(N)} should 
contain $(s^0_n)_1$, $(s^0_n)_2$, respectively.
$(G^0(\varphi_j))_1$ and 
$(G^0(\varphi_j))_2$ 
are to be contained in \texttt{G1(J)} and \texttt{G2(J)}, respectively.

(b) \texttt{G1, G2}
must be aligned with 32byte boundary
if you did make ISPACK with setting either SSE=avx or SSE=fma
on Intel x86 CPUs and setting \texttt{JM} to be a multiple of 8.
The arrays \texttt{G1, G2} must be aligned with 64byte boundary
if you did make ISPACK with setting SSE=avx512
on Intel x86 CPUs and setting \texttt{JM} to be a multiple of 16.

(c) If you set \texttt{IPOW}$=l$, 
$(\cos\varphi_j)^{-l}G^0(\varphi_j)_{(1,2)}$ are returned 
in the output arrays \texttt{G1, G2} instead of $G^0(\varphi_j)_{(1,2)}$.

\end{enumerate}

%---------------------------------------------------------------------

\subsection{LXTVZS}

\begin{enumerate}

\item Purpose

Paired transform from two wave data to two spectral data (zonal componet).

\item Definition

For the zonal component, transform two wave data ($G^0(\varphi_j)$)$_{(1,2)}$
to two corresponding spectral data ($s^0_n$)$_{(1,2)}$ at each latitude
by the backward associated Legendre function transform (see Outline).

\item Synopsis 

\texttt{LXTVZS(NM,NN,JM,S1,S2,G1,G2,PZ,RM,IPOW)}

\item Parameters

(Since most of parameters are the same as in LXTSZV,
the following explanation is only for parameters
different from those in LXTSZV)
  
\begin{tabular}{ll}
\texttt{S1} & Output. Array to contain $(s^0_n)_1$.\\
\texttt{S2} & Outpot. Array to contain $(s^0_n)_2$.\\
\texttt{G1} & Output. Array that contains $(G^0(\varphi_j))_1$.\\
\texttt{G2} & Output. Array that contains $(G^0(\varphi_j))_2$.\\
\texttt{IPOW} & Input. 
The degree of $1/\cos\varphi$ multiplied 
simultaneously.\\
\end{tabular}

\item Pemark

(a) \texttt{S1(N)} and \texttt{S2(N)} are to contained 
$(s^0_n)_1$, $(s^0_n)_2$, respectively.
$(G^0(\varphi_j))_1$ and 
$(G^0(\varphi_j))_2$ 
should be contained in \texttt{G1(J)} and \texttt{G2(J)}, respectively.
  
(b) \texttt{G1, G2}
must be aligned with 32byte boundary
if you did make ISPACK with setting either SSE=avx or SSE=fma
on Intel x86 CPUs and setting \texttt{JM} to be a multiple of 8.
The arrays \texttt{G1, G2} must be aligned with 64byte boundary
if you did make ISPACK with setting SSE=avx512
on Intel x86 CPUs and setting \texttt{JM} to be a multiple of 16.

(c) If you set \texttt{IPOW}$=l$, 
$(\cos\varphi_j)^{-l}G^0(\varphi_j)_{(1,2)}$ are used
as the input instead of $G^0(\varphi_j)_{(1,2)}$.

\end{enumerate}

%---------------------------------------------------------------------
\subsection{LXINIC}

\begin{enumerate}

\item Purpose  

An initialization routine for \texttt{LXPACK}.  
It initializes the array, \texttt{CM},
which is used in other subroutines such as \texttt{LXCSZY}, \texttt{LXCSWY}.
  
\item Definition

\item Synopsis 
  
    
\texttt{LXINIC(NT,M,CM)}
  
\item Parameters

\begin{verbatim}
INTEGER(8) :: NT, M
REAL(8) :: CM(2*(NT-M)+1)
\end{verbatim}
      
\begin{tabular}{ll}
\texttt{NT} & Input. The truncation wavenumber $N_T$
  (see \texttt{LXCSZY}).\\
\texttt{M} & Input. zonal wavenumber $m$ 
(\texttt{M}$\ge$ 0 must hold).\\
\texttt{CM}  & Output. An array which is used in \texttt{LXCSZY},
\texttt{LXCSWY} etc.
\end{tabular}

\item Remark

\end{enumerate}

%---------------------------------------------------------------------
\subsection{LXCSWY}

\begin{enumerate}

\item Purpose 

Conversion of spectral data for backward transform
corresponding to latitudinal derivative (wave component).

\item Definition

If the backward spectral transform, where the 
truncation wavenumber $N_T$ is assumed, is defined as,
\begin{equation}
G^m(\varphi)\equiv\sum^{N_T}_{n=|m|}s^m_nP^m_n(\sin\varphi)
\end{equation}
operating $\cos\varphi\frac{\partial}{\partial\varphi}$ to
$G^m$ yields
\begin{equation}
\cos\varphi\frac{\partial}{\partial\varphi}
G^m(\varphi)=\sum^{N_T+1}_{n=|m|}
(s_y)^m_nP^m_n(\sin\varphi)
\end{equation}
That is, it can be represented by a backward spectral transform
with setting the truncation wavenumber of $n$ as $N_T+1$.
This subroutine computes $(s_y)^m_n$ from $s^m_n$.

\item Synopsis 
    
\texttt{LXCSWY(NT,M,S,SY,CM)}
  
\item Parameter

\begin{verbatim}
INTEGER(8) :: NT, M
REAL(8) :: S(2,M:NT), SY(2,M:NT+1), CM(2*(NT-M)+1)
\end{verbatim}
      
\begin{tabular}{ll}
\texttt{NT} & Input. Truncation wavenumber $N_T$.\\
\texttt{M} & Input. zonal wavenumber $m$ 
(\texttt{M}$\ge$ 0 must hold).\\
\texttt{S} & Input. Array that contains $s^m_n$.\\
\texttt{SY} & Output. Array to contain $(s_y)^m_n$.\\
\texttt{CM}  & Input. The array initialized by \texttt{LXINIC}
with specifying \texttt{M}.
\end{tabular}

\item Remark

(a) If you call \texttt{LXCSWY(NT,M,S,SY,CM)} and 
  \texttt{LXTSWG(NM,NT+1,JM,M,SY,G,PZ,PM,RM,JC,1\_8)}
 continuously, the latitudinal derivative is returned in \texttt{G}.  
   
\end{enumerate}

%---------------------------------------------------------------------
\subsection{LXCYWS}

\begin{enumerate}

\item Purpose 

Conversion of spectral data for forward transform
corresponding to latitudinal derivative (wave component).

\item Definition

If you want to compute the following form of modified forward
spectral transform
(it appers in computing the divergence of a vector field, for example):
\begin{equation}
s^m_n=\frac1{2}\int^{\pi/2}_{-\pi/2}
G^m(\varphi)
\left(-\cos\varphi\frac{\partial}{\partial\varphi}P^m_n(\sin\varphi)\right)
\cos\varphi d\varphi,
\end{equation}
$s^m_n$ can be obtained from the result of 
the following normal forward spectral
transform of $G^m$
\begin{equation}
(s_y)^m_n=\frac1{2}\int^{\pi/2}_{-\pi/2}
G^m(\varphi)
P^m_n(\sin\varphi)
e^{-im\lambda}\cos\varphi d\varphi.
\end{equation}
This subroutine computes $s^m_n$ from $(s_y)^m_n$.
Here, if $s^m_n$ is to be computed to $n=N_T$, $(s_y)^m_n$
must be computed up to  $n=N_T+1$.

\item Synopsis
    
\texttt{LXCYWS(NT,M,SY,S,CM)}
  
\item Parameter

\begin{verbatim}
INTEGER(8) :: NT, M
REAL(8) :: S(2,M:NT), SY(2,M:NT+1), CM(2*(NT-M)+1)
\end{verbatim}
      
\begin{tabular}{ll}
\texttt{NT} & Input. Truncation wavenumber $N_T$.\\
\texttt{M} & Input. zonal wavenumber $m$ 
(\texttt{M}$\ge$ 0 must hold).\\
\texttt{SY} & Input. Array that contains $(s_y)^m_n$.\\
\texttt{S} & Output. Array to contain $s^m_n$.\\
\texttt{CM}  & Input. The array initialized by \texttt{LXINIC}
with specifying \texttt{M}.
\end{tabular}

\item Remark

(a) If you call \texttt{LXTGWS(NM,NT+1,JM,M,SY,G,PZ,PM,RM,JC,1\_8)} and
  \texttt{LXCYWS(NT,M,SY,S,CM)}  continuously,
    the following computation that corresponds to
    the forward transform of the divergence of the latitudinal
    component of a vector field,  
\begin{equation}
s^m_n=\frac1{2}\int^{\pi/2}_{-\pi/2}
\frac{\partial}{\cos\varphi\partial\varphi}
\left(\cos\varphi G^m(\varphi)\right)
P^m_n(\sin\varphi)\cos\varphi d\varphi
\end{equation}
is done and $s^m_n$ is returned in \texttt{S}.
Note that integration by parts is applied and \texttt{IPOW=1} here.
  
\end{enumerate}

%---------------------------------------------------------------------
\subsection{LXCSZY}

\begin{enumerate}

\item Purpose 

Conversion of spectral data for backward transform
corresponding to latitudinal derivative (zonal component).

\item Definition
  

See \texttt{LXCSWY} with setting $m=0$.

\item Synopsis
    
\texttt{LXCSZY(NT,S,SY,CM)}
  
\item Parameter

\begin{verbatim}
INTEGER(8) :: NT
REAL(8) :: S(0:NT), SY(0:NT+1), CM(2*NT+1)
\end{verbatim}

\begin{tabular}{ll}
\texttt{NT} & Input. Truncation wavenumber $N_T$.\\
\texttt{S} & Input. Array that contains $s^0_n$.\\
\texttt{SY} & Output. Array to contain $(s_y)^0_n$.\\
\texttt{CM}  & Input. The array initialized by \texttt{LXINIC}
with specifying \texttt{M=0}.
\end{tabular}

\item Remark

(a) If you call \texttt{LXCSZY(NT,S,SY,CM)} and 
  \texttt{LXTSZG(NM,NT+1,JM,SY,G,PZ,RM,1\_8)}
 continuously, the latitudinal derivative is returned in \texttt{G}.    

\end{enumerate}

%---------------------------------------------------------------------
\subsection{LXCYZS}

\begin{enumerate}

\item Purpose 

Conversion of spectral data for forward transform
corresponding to latitudinal derivative (zonal component).

\item Definition
  
See \texttt{LXCYWS} with setting $m=0$.

\item Synopsis
    
\texttt{LXCYZS(NT,SY,S,CM)}
  
\item Parameter

\begin{verbatim}
INTEGER(8) :: NT
REAL(8) :: S(0:NT), SY(0:NT+1), CM(2*NT+1)
\end{verbatim}
      
\begin{tabular}{ll}
\texttt{NT} & Input. Truncation wavenumber $N_T$.\\
\texttt{SY} & Input. Array that contains $(s_y)^0_n$.\\
\texttt{S} & Output. Array to contain $s^0_n$.\\
\texttt{CM}  & Input. The array initialized by \texttt{LXINIC}
with specifying \texttt{M=0}.
\end{tabular}

\item Remark

(a) If you call \texttt{LXTGZS(NM,NT+1,JM,SY,G,PZ,RM,1\_8)} and 
    \texttt{LXCYZS(NT,SY,S,CM)} continuously,
    the following computation that corresponds to
    the forward transform of the divergence of the latitudinal
    component of a vector field,  
\begin{equation}
s^0_n=\frac1{2}\int^{\pi/2}_{-\pi/2}
\frac{\partial}{\cos\varphi\partial\varphi}
\left(\cos\varphi G^0(\varphi)\right)
P^0_n(\sin\varphi)\cos\varphi d\varphi
\end{equation}
is done and $s^0_n$ is returned in \texttt{S}.
Note that integration by parts is applied and \texttt{IPOW=1} here.
  
\end{enumerate}

%---------------------------------------------------------------------

\subsection{LXTSZP}

\begin{enumerate}

\item Purpose

Transform from zonal spectral data to the grid
values at the poles.
  
\item Definition

For the zonal component, transform spectral data ($s^0_n$) 
to wave data at the north and the south pole, 
$(G^0(\pi/2))$ and $(G^0(-\pi/2))$, respectively
by the backward associated Legendre function transform (see Outline).

\item Synopsis 

\texttt{LXTSZP(NN,S,GNP,GSP)}
  
\item Parameters

\begin{verbatim}  
INTEGER(8) :: NN
REAL(8) ::  S(0:NN), GNP, GSP
\end{verbatim}  

\begin{tabular}{ll}
\texttt{NM} & Input. Maximum value of $N$ to be used.\\
\texttt{S} & Input. Array that contains $s^0_n$.\\
\texttt{GNP} & Output. Variable to contain $G^0(\pi/2)$.\\
\texttt{GSP} & Output. Variable to contain $G^0(-\pi/2)$.
\end{tabular}

\item Remark

\end{enumerate}


\end{document}
