%***********************************************************************
% ISPACK FORTRAN SUBROUTINE LIBRARY FOR SCIENTIFIC COMPUTING
% Copyright (C) 1998--2019 Keiichi Ishioka <ishioka@gfd-dennou.org>
%
% This library is free software; you can redistribute it and/or
% modify it under the terms of the GNU Lesser General Public
% License as published by the Free Software Foundation; either
% version 2.1 of the License, or (at your option) any later version.
%
% This library is distributed in the hope that it will be useful,
% but WITHOUT ANY WARRANTY; without even the implied warranty of
% MERCHANTABILITY or FITNESS FOR A PARTICULAR PURPOSE.  See the GNU
% Lesser General Public License for more details.
% 
% You should have received a copy of the GNU Lesser General Public
% License along with this library; if not, write to the Free Software
% Foundation, Inc., 51 Franklin Street, Fifth Floor, Boston, MA
% 02110-1301 USA.
%***********************************************************************
\documentclass[a4j]{jsarticle}

\title{SYPACK使用の手引}
\author{}
\date{}

\begin{document}

\maketitle

\section{概要}

これは, スペクトル(球面調和関数)変換を行なうサブルーチンパッケージ
SXPACKをMPIで並列化したものである.
従って, 基本的な変換の定義等については
SXPACK使用の手引を参照すること. また, 内部で SXPACKのサブルーチンを
利用している.

本サブルーチンパッケージは内部でMPIのサブルーチンをCALLしているので,
本サブルーチンパッケージに属するサブルーチンをCALLする場合は
\texttt{MPI\_INIT}と\texttt{MPI\_FINALIZE}で囲まれた枠組の中である必要がある.
また, もちろんヘッダファイル(mpif.h)の読み込みもされている必要
がある. これらMPIの一般的な使用法に関しては当該システムにおける
MPIのマニュアルを参照のこと.

\subsection{計算の並列化について}

SYPACKにおけるMPI並列化は, スタンダードなもので, スペクトルデータは東西波
数方向に分割, グリッドデータは緯度方向に分割して各プロセスに格納されて
扱われる. 各プロセスに格納されるスペクトルデータの東西波数および
グリッドデータの緯度についての情報を取得するためのサブルーチンが提供
されている. なお, 各プロセス内での OpenMP並列化もなされている.

\textbf{以下のサブルーチンの説明において, MPIのプロセス数を, 整数型変数
\texttt{NP} として記述する.}


%%%%%%%%%%%%%%%%%%%%%%%%%%%%%%%%%%%%%%%%%%%%%%%%%%%%%%%%%%%%%%%%%%%%%%

\section{サブルーチンのリスト}

\vspace{1em}
\begin{tabular}{ll}
\texttt{SYINI1} & 初期化その1\\
\texttt{SYINI2} & 初期化その2\\  
\texttt{SYNM2L} & スペクトルデータの格納位置の計算\\
\texttt{SYL2NM} & \texttt{SYNM2L}の逆演算\\
\texttt{SYQRNM} & そのプロセスが担当する東西波数の問い合わせ\\
\texttt{SYQRNJ} & そのプロセスが担当するガウス緯度の問い合わせ\\
\texttt{SYTS2G} & スペクトルデータからグリッドデータへの変換\\
\texttt{SYTG2S} & グリッドデータからスペクトルデータへの変換\\
\texttt{SYTS2V} & スペクトルデータからグリッドデータへの2個同時変換\\
\texttt{SYTV2S} & グリッドデータからスペクトルデータへの2個同時変換\\
\texttt{SYGS2S} & 分散配置されたスペクトルデータの集約\\
\texttt{SYSS2S} & 集約されたスペクトルデータの分散\\
\texttt{SYGG2G} & 分散配置されたグリッドデータの集約\\
\texttt{SYSG2G} & 集約されたグリッドデータの分散\\
\texttt{SYINIC} & \texttt{SYCS2Y}, \texttt{SYCY2S} 等で用いられる配列の初期化\\
\texttt{SYCS2Y} & 緯度微分を作用させた逆変換に対応する
スペクトルデータの変換\\
\texttt{SYCY2S} & 緯度微分を作用させた正変換に対応する
スペクトルデータの変換\\
\texttt{SYCS2X} & 経度微分に対応するスペクトルデータの変換\\
\texttt{SYINID} & \texttt{SYCLAP} で用いられる配列の初期化\\
\texttt{SYCLAP} & スペクトルデータにラプラシアンを作用, またはその逆演算\\
\texttt{SYCRPK} & 切断波数の異なるスペクトルデータの詰め替え\\
\texttt{SYKINI} & 多層用初期化\\
\texttt{SYKGXX} & 多層用のデータ集約\\
\texttt{SYKSXX} & 多層用のデータ分散\\
\texttt{SYQRJV} & \texttt{JV}の値の推奨値の問い合わせ
\end{tabular}

%%%%%%%%%%%%%%%%%%%%%%%%%%%%%%%%%%%%%%%%%%%%%%%%%%%%%%%%%%%%%%%%%%%%%%

\section{サブルーチンの説明}

\subsection{SYINI1}

\begin{enumerate}

\item 機能
\texttt{SYPACK}の初期化ルーチンその1.
\texttt{SYPACK}の他のサブルーチンで使われる配列\texttt{IT, T, R}
の値を初期化する.

\item 定義

\item 呼び出し方法 
    
\texttt{SYINI1(MM,NM,IM,IT,T,R,ICOM)}
  
\item パラメーターの説明

\begin{verbatim}  
INTEGER(8) :: MM, NM, IM, IT(IM/2), ICOM
REAL(8) :: T(IM*3/2)
REAL(8) :: R(5*(MM/NP+1)*(2*NM-MM/NP*NP)/4+MM/NP+1)
\end{verbatim}  
    
\begin{tabular}{ll}
\texttt{MM} & 入力. $m$の切断波数$M$.\\
\texttt{NM} & 入力. $n$の切断波数$N$の使いうる最大値.\\
\texttt{IM} & 入力. 東西格子点数.\\
\texttt{IT} & 出力. \texttt{SYPACK}の他のルーチンで用いられる配列.\\
\texttt{T}  & 出力. \texttt{SYPACK}の他のルーチンで用いられる配列.\\
\texttt{R}  & 出力. \texttt{SYPACK}の他のルーチンで用いられる配列.\\
\texttt{ICOM} & 入力. 計算に用いる MPIコミュニケーター.
\end{tabular}

\item 備考

(a) \texttt{MM} は 0以上の整数であること.
  また, 通常の三角切断の場合で, かつ緯度方向の微分演算
  (\texttt{SYCS2Y}等を参照)
     を行う場合, 三角切断の切断波数を \texttt{NT} とすると,
     \texttt{MM=NT},  \texttt{NM=NT+1} と定めれば良い.
     
(b) \texttt{IM} は \texttt{IM > 2*MM}を満し, かつ\texttt{IM/2}が
2, 3, 5で素因数分解できる整数でなければならない.

(c) \texttt{NM} は \texttt{NM} $\ge$ \texttt{MM}を満していなければなら
ない.

(d) \texttt{ICOM}は計算に用いる MPIコミュニケーターを指定する.
MPI の Fortranでの実装では, 通常, MPIコミュニケーターは
\texttt{INTEGER(4)}型となっているので, 適宜\texttt{INTEGER(8)}に
キャストして入力とすること.

(e) 配列\texttt{R}の大きさを定めるのに用いられる\texttt{NP}は
\texttt{ICOM}で指定されるコミュニケーターに含まれるプロセス数(サイズ)
である. 配列\texttt{R}の領域を動的に確保しておく場合には
コミュニケーター\texttt{ICOM}のサイズを\texttt{MPI\_COMM\_SIZE}で
取得してそれを\texttt{NP}とすれば良いが, 静的に確保する場合は,
設定しうるサイズの最小値を\texttt{NP}としておけばよい.

\end{enumerate}

%---------------------------------------------------------------------

\subsection{SYINI2}

\begin{enumerate}

\item 機能
\texttt{SYPACK}の初期化ルーチンその2.
\texttt{SYPACK}の他のサブルーチンで使われる配列\texttt{P, JC}
の値を初期化する.

\item 定義

\item 呼び出し方法 
    
\texttt{SYINI2(MM,NM,JM,IG,P,R,JC,ICOM)}
  
\item パラメーターの説明

\begin{verbatim}  
INTEGER(8) :: MM, NM, JM, IG, JC((MM/NP+1)*(2*NM-MM/NP*NP)/16+MM/NP+1)
REAL(8) :: P(JM/2,5+2*(MM/NP+1))
REAL(8) :: R(5*(MM/NP+1)*(2*NM-MM/NP*NP)/4+MM/NP+1)
\end{verbatim}  
    
\begin{tabular}{ll}
\texttt{MM} & 入力. $m$の切断波数$M$.\\
\texttt{NM} & 入力. $n$の切断波数$N$の使いうる最大値.\\
\texttt{JM} & 入力. 南北格子点数.\\
\texttt{IG} & 入力. グリッドの種類(備考参照).\\
\texttt{P}  & 出力. \texttt{SYPACK}の他のルーチンで用いられる配列.\\
\texttt{R}  & 入力. \texttt{SYINI1}で初期化された配列.\\
\texttt{JC}  & 出力. \texttt{SYPACK}の他のルーチンで用いられる配列.\\
\texttt{ICOM} & 入力. 計算に用いる MPIコミュニケーター.
\end{tabular}

\item 備考

(a) \texttt{MM} は 0以上の整数であること.
     また, 通常の三角切断の場合で, かつ緯度方向の微分演算
     (\texttt{SYCS2Y}等を参照)
     を行う場合, 三角切断の切断波数を \texttt{NT} とすると,
     \texttt{MM=NT},  \texttt{NM=NT+1} と定めれば良い.
     
\texttt{JM}は 2以上の偶数であること. さらに, ISPACKのインストール時に
SSE=avx または SSE=fma とした場合は 8の倍数にしておくと高速になる.
また, SSE=avx512 とした場合は 16の倍数にしておくと高速になる.

\texttt{NM} は \texttt{NM} $\ge$ \texttt{MM}を満していなければならない.

(b) \texttt{IG=1}の場合は, 緯度方向分点$(\varphi_j)$や対応する重み$(w_j)$と
して, 通常のガウス緯度とガウス重みが設定される.
\texttt{IG=2}とすると, 緯度方向分点は, 
緯度方向を\texttt{JM+1}等分して両極を除いた\texttt{JM}個の分点
が設定され, 対応する重みは, Clenshaw-Curtis求積法に用いられる重みが設定される.
\texttt{IG=3}とすると, 緯度方向分点は, 
緯度方向を\texttt{JM}等分して両極を除いた\texttt{JM-1}個の緯度
方向分点が設定される. ただし, 赤道上の分点は変換ルーチンにおいて
重複して扱われる.
対応する重みは, Clenshaw-Curtis求積法に用いられる重みが設定される.
この場合, 変換ルーチンにおいて
赤道の分点が重複して扱われるので, 対応する重みは
Clenshaw-Curtis求積法の重みの半分に設定される.

変換ルーチンで用いるスペクトルデータの切断波数が
\texttt{NM}の場合, スペクトルデータから逆変換した
後に正変換を行って(丸め誤差を除いて)元に戻るためには,
\texttt{IG=1}の場合は, \texttt{JM}$\ge$ \texttt{NM+1}が,
\texttt{IG=2}の場合は, \texttt{JM}$\ge$ \texttt{2*NM+1}が,
\texttt{IG=3}の場合は, \texttt{JM}$\ge$ \texttt{2*(NM+1)}が,
それぞれ満されていなければならない.

(c) 配列\texttt{P}が\texttt{P(JM/2,5+2*(MM/NP+1))}と宣言されている場合, 
   \texttt{P(J,1)}:  $\sin(\varphi_{J/2+j})$,
   \texttt{P(J,2)}:  $\frac12 w_{J/2+j}$, 
   \texttt{P(J,3)}:  $\cos(\varphi_{J/2+j})$,
   \texttt{P(J,4)}:  $1/\cos(\varphi_{J/2+j})$,
   \texttt{P(J,5)}:  $\sin^2(\varphi_{J/2+j})$,
が格納される(ガウス緯度 $\varphi_j$に関する説明は SXPACKのマニュアル参
照).
. また, 配列 \texttt{P}の先頭アドレスは, 
ISPACKのインストール時に SSE=avx または SSE=fma とし, かつ
JMを8の倍数とした場合は, 32バイト境界に
合っていなければならない.
また, SSE=avx512 とし, かつ
JMを 16の倍数とした場合は, 64バイト境界に
合っていなければならない.

(d) コミュニケーター\texttt{ICOM}および配列\texttt{JC}の
大きさを定めるのに用いられる\texttt{NP}に関しては\texttt{SYINI1}の
項を参照.


\end{enumerate}

%---------------------------------------------------------------------

\subsection{SYNM2L}

\begin{enumerate}

\item 機能 

全波数と帯状波数から担当プロセス番号とスペクトルデータの格納位置を計算する.  

\item 定義

SYPACKにおいて, スペクトルデータは, 各プロセスで分散して扱われる.
このサブルーチンは, スペクトルデータ$s^m_n$の全波数$n$, 
および帯状波数$m$か
ら, 担当するプロセスのID(ランク)と配列中の格納位置を求めるものである.

\item 呼び出し方法 
    
\texttt{SYNM2L(NN,N,M,IP,L,ICOM)}
  
\item パラメーターの説明

\begin{verbatim}
INTEGER(8) :: NN, N, M, IP, L, ICOM
\end{verbatim}    

\begin{tabular}{ll}
\texttt{NN} & 入力. $n$の切断波数$N$.\\
\texttt{N} & 入力. 全波数.\\
\texttt{M} & 入力. 帯状波数(備考参照).\\
\texttt{IP} & 出力. 担当するプロセスのID(ランク).\\
\texttt{L} & 出力. スペクトルデータの格納位置.\\
\texttt{ICOM} & 入力. 計算に用いる MPIコミュニケーター.
\end{tabular}

\item 備考

\texttt{M} $>$ 0 なら $m=$ \texttt{M}, $n=$ \texttt{N}として$\mbox{Re}(s^m_n)$の格納
位置を, \texttt{M} $<$ 0 なら $m=$ \texttt{-M}, $n=$ \texttt{N}として
$\mbox{Im}(s^m_n)$の格納位置を返す.

\end{enumerate}

%---------------------------------------------------------------------

\subsection{SYL2NM}

\begin{enumerate}

\item 機能 
\texttt{SYNM2L}の逆演算, すなわち, 各プロセスにおける
スペクトルデータの格納位置から全波数と帯状波数を求める.

\item 定義

\texttt{SYNM2L}の項を参照

\item 呼び出し方法 
    
\texttt{SYL2NM(MM,NN,L,N,M,ICOM)}
  
\item パラメーターの説明

\begin{verbatim}
INTEGER(8) :: MM, NN, N, M, L, ICOM
\end{verbatim}    

\begin{tabular}{ll}
\texttt{MM} & 入力. $m$の切断波数$M$.\\  
\texttt{NN} & 入力. $n$の切断波数$N$.\\
\texttt{L} & 入力. スペクトルデータの格納位置.\\
\texttt{N} & 出力. 全波数.\\
\texttt{M} & 出力. 帯状波数.\\
\texttt{ICOM} & 入力. 計算に用いる MPIコミュニケーター.
\end{tabular}

\item 備考

(a)  \texttt{M} の正負についての意味づけは\texttt{SYNM2L}と同じである.

(b) そのプロセスでスペクトルデータが扱われない場合, または \texttt{L}
    で指定された位置にスペクトルデータが存在しない場合には, \texttt{N=-1}, 
    \texttt{M=0} が出力として返される.

\end{enumerate}

%---------------------------------------------------------------------

\subsection{SYQRNM}

\begin{enumerate}

\item 機能 

そのプロセスが担当する東西波数の問い合わせ

\item 定義

スペクトルデータのうち, そのプロセスが担当する東西波数を返す.

\item 呼び出し方法 
    
\texttt{SYQRNM(MM,MCM,MC,ICOM)}
  
\item パラメーターの説明

\begin{verbatim}
INTEGER(8) :: MM, MCM, MC(MM/NP+1), ICOM
\end{verbatim}    

\begin{tabular}{lll}
\texttt{MM} & 入力. $m$の切断波数$M$.\\
\texttt{MCM} & 出力. そのプロセスが担当する$m$の個数.\\
\texttt{MC} & 出力. そのプロセスが担当する$m$が格納される配列.\\
\texttt{ICOM} & 入力. 計算に用いる MPIコミュニケーター.
\end{tabular}

\item 備考

(a) そのプロセスがスペクトルデータを扱わない場合, \texttt{MCM=0}が返され
  る.

(b) \texttt{MCM(1:MCM)}に, そのプロセスが扱う$m$が格納される.

\end{enumerate}

%---------------------------------------------------------------------

\subsection{SYQRNJ}

\begin{enumerate}

\item 機能 

そのプロセスが担当するガウス緯度の問い合わせ

\item 定義

グリッドデータのうち, そのプロセスが担当するガウス緯度の範囲を返す.

\item 呼び出し方法 
    
\texttt{SYQRNJ(JM,JV,J1,J2,ICOM)}
  
\item パラメーターの説明

\begin{verbatim}
INTEGER(8) :: JM, JV, J1, J2, ICOM
\end{verbatim}    

\begin{tabular}{ll}
\texttt{JM} & 入力. 南北格子点数.\\
\texttt{JV} & 入力. 変換のベクトル長.\\
\texttt{J1} & 出力. そのプロセスが担当する$j$の下端.\\
\texttt{J2} & 出力. そのプロセスが担当する$j$の上端.\\
\texttt{ICOM} & 入力. 計算に用いる MPIコミュニケーター.
\end{tabular}

\item 備考

(a) そのプロセスがグリッドデータを扱わない場合, \texttt{J1=0}および
\texttt{J2=-1}が返される.

\end{enumerate}

%---------------------------------------------------------------------

\subsection{SYTS2G}

\begin{enumerate}

\item 機能 

スペクトルデータからグリッドデータへの変換を行う.

\item 定義

スペクトル逆変換により分散配置されたスペクトルデータ($s^m_n$)
から分散配置されたグリッドデータ($g(\lambda_i,\varphi_j)$)を求める.

\item 呼び出し方法 

\texttt{SYTS2G(MM,NM,NN,IM,JM,JV,S,G,IT,T,P,R,JC,W,IPOW,ICOM)}
  
\item パラメーターの説明

\begin{verbatim}        
INTEGER(8) :: MM, NM, NN, IM, JM, JV, IT(IM/2)
INTEGER(8) :: JC((MM/NP+1)*(2*NM-MM/NP*NP)/16+MM/NP+1), IPOW, ICOM
REAL(8) :: S((MM/NP+1)*(2*(NN+1)-MM/NP*NP)), G(0:IM-1,((JM/JV-1)/NP+1)*JV)
REAL(8) :: T(IM*3/2), P(JM/2,5+2*(MM/NP+1)), R(5*(MM/NP+1)*(2*NM-MM/NP*NP)/4+MM/NP+1)
REAL(8) :: W(2*JV*((JM/JV-1)/NP+1)*(MM/NP+1)*NP*2)
\end{verbatim}      

\begin{tabular}{ll}
\texttt{MM} & 入力. $m$の切断波数.\\
\texttt{NM} & 入力. $n$の切断波数の最大値.\\
\texttt{NN} & 入力. $n$の切断波数.
(\texttt{MM}$\le$\texttt{NN}$\le$\texttt{NM}であること)\\
\texttt{IM} & 入力. 東西格子点数.\\
\texttt{JM} & 入力. 南北格子点数.\\
\texttt{JV} & 入力. 変換のベクトル長.\\
\texttt{S} & 入力. $s^m_n$が格納されている配列.\\
\texttt{G} & 出力. $g(\lambda_i,\varphi_j)$が格納される配列.\\
\texttt{IT} & 入力. \texttt{SYINI1}で初期化された配列.\\
\texttt{T} & 入力. \texttt{SYINI1}で初期化された配列.\\
\texttt{P}  & 入力. \texttt{SYINI2}で初期化された配列.\\
\texttt{R}  & 入力. \texttt{SYINI1}で初期化された配列.\\
\texttt{JC}  & 入力. \texttt{SYINI2}で初期化された配列.\\
\texttt{W} & 作業領域.\\
\texttt{IPOW} & 入力. 逆変換と同時に作用させる
                      $1/\cos\varphi$の次数. 0から2までの整数.\\
\texttt{ICOM} & 入力. 計算に用いる MPIコミュニケーター.
\end{tabular}

\item 備考

(a) \texttt{JV}は \texttt{JM/2}の約数でなければならない.
ISPACKのインストール時に SSE=avx または SSE=fma としている
場合, \texttt{JV}を 4とすれば高速になる.
また, SSE=avx512 とした場合は 8とすれば高速になる.
  
(b) \texttt{G, W, P}の先頭アドレスは, 
ISPACKのインストール時に SSE=avx または SSE=fma とし, かつ
\texttt{JV}を4とした場合は, 32バイト境界に合っていなければならない.
また, SSE=avx512 とし, かつ
\texttt{JV}を8とした場合は, 64バイト境界に合っていなければならない.

(c) \texttt{G(0:IM-1,((JM/JV-1)/NP+1)*JV)}と宣言されている場合, 
    \texttt{G(I,J-J1+1)}には
    $g(\lambda_i,\varphi_j)$が格納される(\texttt{J}は\texttt{J1}
    から\texttt{J2}まで.
    ここに, \texttt{J1}, \texttt{J2}, は \texttt{SYQRNJ}で与えられる
    そのプロセスが担当する\texttt{J}の範囲の下端と上端).

(d) \texttt{IPOW}$=l$とすると, $g(\lambda_i,\varphi_j)$の
    かわりに $(\cos\varphi_j)^{-l}g(\lambda_i,\varphi_j)$ が出力される.

\end{enumerate}

%---------------------------------------------------------------------

\subsection{SYTG2S}

\begin{enumerate}

\item 機能 

グリッドデータからスペクトルデータへの変換を行う.

\item 定義

スペクトル正変換により分散配置されたグリッドデータ($g(\lambda_i,\varphi_j)$)
から分散配置されたスペクトルデータ($s^m_n$)を求める.

\item 呼び出し方法 

\texttt{SYTG2S(MM,NM,NN,IM,JM,JV,S,G,IT,T,P,R,JC,W,IPOW,ICOM)}  

\item パラメーターの説明(殆んど \texttt{SYTS2G}の項と同じであるので,
異なる部分のみについて記述する).

\begin{tabular}{ll}
\texttt{S} & 出力 $s^m_n$が格納される配列.\\
\texttt{G} & 入力. $g(\lambda_i,\varphi_j)$が格納されている配列.\\
\texttt{IPOW} & 入力. 正変換と同時に作用させる
                      $1/\cos\varphi$の次数. 0から2までの整数.
\end{tabular}

\item 備考

(a) \texttt{G(I,J)}には
    $g(\lambda_i,\varphi_j)$を格納すること.

(b) \texttt{G(0:IM-1,((JM/JV-1)/NP+1)*JV)}と宣言されている場合, 
    \texttt{G(I,J-J1+1)}には
    $g(\lambda_i,\varphi_j)$を格納していること(\texttt{J}は\texttt{J1}
    から\texttt{J2}まで.
    ここに, \texttt{J1}, \texttt{J2}, は \texttt{SYQRNJ}で与えられる
    そのプロセスが担当する\texttt{J}の範囲の下端と上端).
    
(c) \texttt{IPOW}$=l$とすると, $g(\lambda_i,\varphi_j)$の
    かわりに $(\cos\varphi_j)^{-l}g(\lambda_i,\varphi_j)$ が入力
    になる.


   
\end{enumerate}

%---------------------------------------------------------------------

\subsection{SYTS2V}

\begin{enumerate}

\item 機能

スペクトルデータからグリッドデータへの2個同時変換を行う.

\item 定義

スペクトル逆変換により分散配置された2個のスペクトルデータ$(s^m_n)_{(1,2)}$
から対応する2個の分散配置されたグリッドデータ
$(g(\lambda_i,\varphi_j))_{(1,2)}$を求める.

  
\item 呼び出し方法 

\texttt{SYTS2V(MM,NM,NN,IM,JM,JV,S1,S2,G1,G2,IT,T,P,R,JC,W,IPOW,ICOM)}
  
\item パラメーターの説明

\begin{verbatim}        
INTEGER(8) :: MM, NM, NN, IM, JM, JV, IT(IM/2)
INTEGER(8) :: JC((MM/NP+1)*(2*NM-MM/NP*NP)/16+MM/NP+1), IPOW, ICOM
REAL(8) :: S1((MM/NP+1)*(2*(NN+1)-MM/NP*NP)), S2((MM/NP+1)*(2*(NN+1)-MM/NP*NP))
REAL(8) :: G1(0:IM-1,((JM/JV-1)/NP+1)*JV), G2(0:IM-1,((JM/JV-1)/NP+1)*JV)
REAL(8) :: T(IM*3/2), P(JM/2,5+2*(MM/NP+1)), R(5*(MM/NP+1)*(2*NM-MM/NP*NP)/4+MM/NP+1)
REAL(8) :: W(2*JV*((JM/JV-1)/NP+1)*(MM/NP+1)*NP*2*2)
\end{verbatim}      

\begin{tabular}{ll}
\texttt{MM} & 入力. $m$の切断波数.\\
\texttt{NM} & 入力. $n$の切断波数の最大値.\\
\texttt{NN} & 入力. $n$の切断波数.
(\texttt{MM}$\le$\texttt{NN}$\le$\texttt{NM}であること)\\
\texttt{IM} & 入力. 東西格子点数.\\
\texttt{JM} & 入力. 南北格子点数.\\
\texttt{JV} & 入力. 変換のベクトル長.\\
\texttt{S1} & 入力. $(s^m_n)_1$が格納されている配列.\\
\texttt{S2} & 入力. $(s^m_n)_2$が格納されている配列.\\
\texttt{G1} & 出力. $(g(\lambda_i,\varphi_j))_1$が格納される配列.\\
\texttt{G2} & 出力. $(g(\lambda_i,\varphi_j))_2$が格納される配列.\\
\texttt{IT} & 入力. \texttt{SYINI1}で初期化された配列.\\
\texttt{T} & 入力. \texttt{SYINI1}で初期化された配列.\\
\texttt{P}  & 入力. \texttt{SYINI2}で初期化された配列.\\
\texttt{R}  & 入力. \texttt{SYINI1}で初期化された配列.\\
\texttt{JC}  & 入力. \texttt{SYINI2}で初期化された配列.\\
\texttt{W} & 作業領域.\\
\texttt{IPOW} & 入力. 逆変換と同時に作用させる
                      $1/\cos\varphi$の次数. 0から2までの整数.\\
\texttt{ICOM} & 入力. 計算に用いる MPIコミュニケーター.
\end{tabular}

\item 備考

(a) 作業領域\texttt{W}の大きさが \texttt{SYTS2G}の倍必要である
      ことに注意すること.

(b) \texttt{G1, G2, W, P}の先頭アドレスは, 
ISPACKのインストール時に SSE=avx または SSE=fma とし, かつ
\texttt{JV}を4とした場合は, 32バイト境界に合っていなければならない.
また, SSE=avx512 とし, かつ
\texttt{JV}を8とした場合は, 64バイト境界に合っていなければならない.

(c) その他の詳細は \texttt{SYTS2G}と共通であるので, そちらの項を
参照すること.

\end{enumerate}

%---------------------------------------------------------------------

\subsection{SYTV2S}

\begin{enumerate}

\item 機能

グリッドデータからスペクトルデータへの2個同時変換を行う.

\item 定義

スペクトル正変換により分散配置された2個の
グリッドデータ$(g(\lambda_i,\varphi_j))_{(1,2)}$
から対応する分散配置された2個のスペクトルデータ$(s^m_n)_{(1,2)}$
を求める.

\item 呼び出し方法 

\texttt{SYTV2S(MM,NM,NN,IM,JM,JV,S1,S2,G1,G2,IT,T,P,R,JC,W,IPOW,ICOM)}
  
\item パラメーターの説明

\item パラメーターの説明(殆んど \texttt{SYTS2V}の項と同じであるので,
異なる部分のみについて記述する).
  
\begin{tabular}{ll}
\texttt{S1} & 出力. $(s^m_n)_1$が格納される配列.\\
\texttt{S2} & 出力. $(s^m_n)_2$が格納される配列.\\
\texttt{G1} & 入力. $(g(\lambda_i,\varphi_j))_1$が格納されている配列.\\
\texttt{G2} & 入力. $(g(\lambda_i,\varphi_j))_2$が格納されている配列.\\
\texttt{IPOW} & 入力. 正変換と同時に作用させる
                      $1/\cos\varphi$の次数. 0から2までの整数.
\end{tabular}

\item 備考

(a) 作業領域\texttt{W}の大きさが \texttt{SYTG2S}の倍必要である
      ことに注意すること.

(b) \texttt{G1, G2, W, P}の先頭アドレスは, 
ISPACKのインストール時に SSE=avx または SSE=fma とし, かつ
\texttt{JV}を4とした場合は, 32バイト境界に合っていなければならない.
また, SSE=avx512 とし, かつ
\texttt{JV}を8とした場合は, 64バイト境界に合っていなければならない.

(c) その他の詳細は \texttt{SYTS2V}と共通であるので, そちらの項を
参照すること.

\end{enumerate}

%---------------------------------------------------------------------

\subsection{SYGS2S}

\begin{enumerate}

\item 機能 

分散配置されたスペクトルデータの集約.

\item 定義

各プロセスに分散配置されたスペクトルデータをランク0のプロセスに集約する.

\item 呼び出し方法 

\texttt{SYGS2S(MM,NN,S,SALL,ICOM)}
  
\item パラメーターの説明

\begin{verbatim}        
INTEGER(8) :: MM, NM, ICOM
REAL(8) :: S((MM/NP+1)*(2*(NN+1)-MM/NP*NP)), SALL((2*NN+1-MM)*MM+NN+1)
\end{verbatim}      

\begin{tabular}{lll}
\texttt{MM} & 入力. $m$の切断波数.\\
\texttt{NN} & 入力. $n$の切断波数.\\
\texttt{S} & 入力. 分散配置されたスペクトルデータが格納されている配列.\\
\texttt{SALL} & 出力. 集約されたスペクトルデータが格納される配列.\\
\texttt{ICOM} & 入力. 計算に用いる MPIコミュニケーター.
\end{tabular}

\item 備考

(a) このサブルーチンはすべてのプロセスで CALLされる必要があるが,
\texttt{SALL}の領域は, ランク0のプロセスでのみ確保されていれば良い.

\end{enumerate}


%---------------------------------------------------------------------

\subsection{SYSS2S}

\begin{enumerate}

\item 機能 

集約されたスペクトルデータの分散.

\item 定義

ランク0のプロセスに集約されているスペクトルデータを各プロセスに分散配置する.

\item 呼び出し方法 

\texttt{SYSS2S(MM,NN,SALL,S,ICOM)}
  
\item パラメーターの説明

\begin{verbatim}        
INTEGER(8) :: MM, NM, ICOM
REAL(8) :: S((MM/NP+1)*(2*(NN+1)-MM/NP*NP)), SALL((2*NN+1-MM)*MM+NN+1)
\end{verbatim}      
  

\begin{tabular}{ll}
\texttt{MM} & 入力. $m$の切断波数\\
\texttt{NN} & 入力. $n$の切断波数\\
\texttt{SALL} 
 &  入力. 集約されたスペクトルデータが格納されている配列.\\
\texttt{S} 
&  出力. 分散配置されたスペクトルデータが格納される配列.\\
\texttt{ICOM} & 入力. 計算に用いる MPIコミュニケーター.
\end{tabular}

\item 備考

(a) このサブルーチンはすべてのプロセスで CALLされる必要があるが,
\texttt{SALL}の領域は, ランク0のプロセスでのみ確保されていれば良い.

\end{enumerate}

%---------------------------------------------------------------------

\subsection{SYGG2G}

\begin{enumerate}

\item 機能 

分散配置されたグリッドデータの集約.

\item 定義

各プロセスに分散配置されたグリッドデータをランク0のプロセスに集約する.

\item 呼び出し方法 

\texttt{SYSG2G(IM,JM,JV,GALL,G,ICOM)}
  
\item パラメーターの説明

\begin{verbatim}        
INTEGER(8) :: IM, JM, JV, ICOM
REAL(8) :: GALL(IM*JM), G(IM*((JM/JV-1)/NP+1)*JV)
\end{verbatim}      

\begin{tabular}{ll}
\texttt{IM} & 入力. 東西格子点数.\\
\texttt{JM} & 入力. 南北格子点数.\\
\texttt{JV} & 入力. 変換のベクトル長.\\
\texttt{G} & 入力. 分散配置されたグリッドデータが格納されている配列.\\
\texttt{GALL} &  出力. 集約されたグリッドデータが格納される配列.\\
\texttt{ICOM} & 入力. 計算に用いる MPIコミュニケーター.
\end{tabular}

\item 備考

(a) このサブルーチンはすべてのプロセスで CALLされる必要があるが,
\texttt{GALL}の領域は, ランク0のプロセスでのみ確保されていれば良い.

\end{enumerate}


%---------------------------------------------------------------------

\subsection{SYSG2G}

\begin{enumerate}

\item 機能 

集約されたグリッドデータの分散.

\item 定義

ランク0のプロセスに集約されたグリッドデータを各プロセスに分散配置する.

\item 呼び出し方法 

\texttt{SYSG2G(IM,JM,JV,GALL,G,ICOM)}
  
\item パラメーターの説明

\begin{verbatim}        
INTEGER(8) :: IM, JM, JV, ICOM
REAL(8) :: GALL(IM*JM), G(IM*((JM/JV-1)/NP+1)*JV)
\end{verbatim}      

\begin{tabular}{ll}
\texttt{IM} & 入力. 東西格子点数.\\
\texttt{JM} & 入力. 南北格子点数.\\
\texttt{JV} & 入力. 変換のベクトル長.\\
\texttt{GALL} 
 &  入力. 集約されたグリッドデータが格納されている配列.\\
\texttt{G} 
& 出力. 分散配置されたグリッドデータが格納される配列.\\
\texttt{ICOM} & 入力. 計算に用いる MPIコミュニケーター.
\end{tabular}

\item 備考

(a) このサブルーチンはすべてのプロセスで CALLされる必要があるが,
\texttt{GALL}の領域は, ランク0のプロセスでのみ確保されていれば良い.

\end{enumerate}

%---------------------------------------------------------------------

\subsection{SYINIC}

\begin{enumerate}

\item 機能 

\texttt{SYCS2Y}, \texttt{SYCY2S}で用いられる配列の\texttt{C}の初期化.

\item 定義

\texttt{SYCS2Y}, \texttt{SYCY2S}の項を参照.

\item 呼び出し方法 

\texttt{SYINIC(MM,NT,C,ICOM)}
  
\item パラメーターの説明

\begin{verbatim}
INTEGER(8) :: MM, NT, ICOM
REAL(8) :: C((MM/NP+1)*(2*(NT+1)-MM/NP*NP))
\end{verbatim}

\begin{tabular}{ll}
\texttt{MM} & 入力. 帯状波数の切断波数.\\    
\texttt{NT} & 入力. 全波数の切断波数.\\
\texttt{C} & 出力. \texttt{SYCS2Y}, \texttt{SYCY2S}で用いられる配列.\\
\texttt{ICOM} & 入力. 計算に用いる MPIコミュニケーター.
\end{tabular}

\item 備考

\end{enumerate}

%---------------------------------------------------------------------

\subsection{SYCS2Y}

\begin{enumerate}

\item 機能 

緯度微分を作用させた逆変換に対応するスペクトルデータの変換を行う.

\item 定義

  スペクトル逆変換(ただし, ここでは $n$方向の切断波数を$N_T$とする)が
\begin{equation}
g(\lambda,\varphi)=\sum^M_{m=-M}\sum^{N_T}_{n=|m|}
s^m_nP^m_n(\sin\varphi)e^{im\lambda}
\end{equation}
と定義されているとき, $g$に$\cos\varphi\frac{\partial}{\partial\varphi}$
を作用させた結果は
\begin{equation}
\cos\varphi\frac{\partial}{\partial\varphi}
g(\lambda,\varphi)=\sum^M_{m=-M}\sum^{N_T+1}_{n=|m|}
(s_y)^m_nP^m_n(\sin\varphi)e^{im\lambda}
\end{equation}
のように$n$方向の切断波数$N_T+1$のスペクトル逆変換で表せる. この
サブルーチンは, $s^m_n$から$(s_y)^m_n$を求めるものである.

\item 呼び出し方法 

\texttt{SYCS2Y(MM,NT,S,SY,C,ICOM)}
  
\item パラメーターの説明

\begin{verbatim}
INTEGER(8) :: MM, NT, ICOM
REAL(8) :: S((MM/NP+1)*(2*(NT+1)-MM/NP*NP)), SY((MM/NP+1)*(2*(NT+2)-MM/NP*NP))
REAL(8) :: C((MM/NP+1)*(2*(NT+1)-MM/NP*NP))
\end{verbatim}

\begin{tabular}{lll}
\texttt{MM} & 入力. 帯状波数の切断波数.\\    
\texttt{NT} & 入力. 全波数の切断波数.\\
\texttt{S} & 入力. $s^m_n$が格納されている配列.\\
\texttt{SY} & 出力. $(s_y)^m_n$が格納される配列.\\
\texttt{C} & 入力. \texttt{SYINIC}で初期化された配列.\\
\texttt{ICOM} & 入力. 計算に用いる MPIコミュニケーター.
\end{tabular}

\item 備考

(a) \texttt{SYCS2Y(MM,NT,S,SY,C,ICOM)} と 
  \texttt{SYTS2G(MM,NM,NN,IM,JM,JV,S,G,IT,T,P,R,JC,W,1\_8,ICOM)}
    とを連続して CALL すれば, \texttt{G}に緯度微分が返されることになる.
    
\end{enumerate}

%---------------------------------------------------------------------

\subsection{SYCY2S}

\begin{enumerate}

\item 機能 

緯度微分を作用させた正変換に対応するスペクトルデータの変換を行う.

\item 定義

以下のような変形されたスペクトル正変換
\begin{equation}
s^m_n=\frac1{4\pi}\int^{2\pi}_0\int^{\pi/2}_{-\pi/2}
g(\lambda,\varphi)
\left(-\cos\varphi\frac{\partial}{\partial\varphi}P^m_n(\sin\varphi)\right)
e^{-im\lambda}\cos\varphi d\varphi
d\lambda .
\quad 
\end{equation}
を計算したい場合(これは, ベクトル場の発散の計算などで現れる), 
$g$の通常のスペクトル正変換
\begin{equation}
(s_y)^m_n=\frac1{4\pi}\int^{2\pi}_0\int^{\pi/2}_{-\pi/2}
g(\lambda,\varphi)
P^m_n(\sin\varphi)
e^{-im\lambda}\cos\varphi d\varphi
d\lambda
\end{equation}
の結果から$s^m_n$を求めることができる.
ただし, $s^m_n$の $n$を $n=N_T$まで求める場合, $(s_y)^m_n$
は $n=N_T+1$まで求めておく必要がある.
このサブルーチンは, $(s_y)^m_n$から$s^m_n$を求めるものである.

\item 呼び出し方法 

\texttt{SYCY2S(MM,NT,SY,S,C,ICOM)}
  
\item パラメーターの説明

\begin{verbatim}
INTEGER(8) :: MM, NT, ICOM
REAL(8) :: S((MM/NP+1)*(2*(NT+1)-MM/NP*NP)), SY((MM/NP+1)*(2*(NT+2)-MM/NP*NP))
REAL(8) :: C((MM/NP+1)*(2*(NT+1)-MM/NP*NP))
\end{verbatim}

\begin{tabular}{lll}
\texttt{MM} & 入力. 帯状波数の切断波数.\\  
\texttt{NT} & 入力. 全波数の切断波数.\\
\texttt{SY} & 入力. $(s_y)^m_n$が格納されている配列.\\
\texttt{S} &  出力. $s^m_n$が格納される配列.\\
\texttt{C} & 入力. \texttt{SYINIC}で初期化された配列.\\
\texttt{ICOM} & 入力. 計算に用いる MPIコミュニケーター.
\end{tabular}

\item 備考

  (a) \texttt{SYTG2S(MM,NM,NN,IM,JM,JV,S,G,IT,T,P,R,JC,W,1\_8,ICOM)}と
    \texttt{SYCY2S(MM,NT,SY,S,C,ICOM)} とを連続して CALL すれば,
    ベクトル場の緯度方向成分の発散の正変換に相当する計算
\begin{equation}
s^m_n=\frac1{4\pi}\int^{2\pi}_0\int^{\pi/2}_{-\pi/2}
\frac{\partial}{\cos\varphi\partial\varphi}
\left(\cos\varphi g(\lambda,\varphi)\right)
P^m_n(\sin\varphi)
e^{-im\lambda}\cos\varphi d\varphi
d\lambda .
\end{equation}
が行われ, \texttt{S}に$s^m_n$が返されることになる
(部分積分していることと, \texttt{IPOW=1} としていることに注意).
    
\end{enumerate}

%---------------------------------------------------------------------

\subsection{SYCS2X}

\begin{enumerate}

\item 機能 

経度微分に対応するスペクトルデータの変換を行う.

\item 定義

スペクトル逆変換(ただし, ここでは $n$方向の切断波数を$N_T$とする)が
\begin{equation}
g(\lambda,\varphi)=\sum^M_{m=-M}\sum^{N_T}_{n=|m|}
s^m_nP^m_n(\sin\varphi)e^{im\lambda}
\end{equation}
と定義されているとき, $g$に$\frac{\partial}{\partial\lambda}$
を作用させた結果は
\begin{equation}
\frac{\partial}{\partial\lambda}
g(\lambda,\varphi)=\sum^M_{m=-M}\sum^{N_T}_{n=|m|}
(s_x)^m_nP^m_n(\sin\varphi)e^{im\lambda}
\end{equation}
のように表せる. ここに, $(s_x)^m_n=ims^m_n$ である.
このサブルーチンは, $s^m_n$から$(s_x)^m_n=ims^m_n$を求めるものである.

\item 呼び出し方法 

\texttt{SYCS2X(MM,NT,S,SX,ICOM)}
  
\item パラメーターの説明

\begin{verbatim}
INTEGER(8) :: MM, NT, ICOM
REAL(8) :: S((MM/NP+1)*(2*(NT+1)-MM/NP*NP)), SX((MM/NP+1)*(2*(NT+1)-MM/NP*NP))
\end{verbatim}

\begin{tabular}{ll}
\texttt{MM} & 入力. 帯状波数の切断波数.\\  
\texttt{NT} & 入力. 全波数の切断波数.\\
\texttt{S} & 入力. $s^m_n$が格納されている配列.\\
\texttt{SX} & 出力 $(s_x)^m_n$が格納される配列.\\
\texttt{ICOM} & 入力. 計算に用いる MPIコミュニケーター.
\end{tabular}

\item 備考

(a) スペクトルデータ $(s_x)^m_n$ の並び順は $s^m_n$ と同じである.
    
\end{enumerate}

%---------------------------------------------------------------------

\subsection{SYINID}

\begin{enumerate}

\item 機能 

\texttt{SYCLAP}で用いられる配列の\texttt{D}の初期化.

\item 定義

\texttt{SYCLAP}の項を参照.

\item 呼び出し方法 

\texttt{SYINID(MM,NT,D,ICOM)}
  
\item パラメーターの説明

\begin{verbatim}
INTEGER(8) :: MM, NT, ICOM
REAL(8) :: D((MM/NP+1)*(2*(NT+1)-MM/NP*NP)*2)
\end{verbatim}

\begin{tabular}{ll}
\texttt{MM} & 入力. 帯状波数の切断波数.\\  
\texttt{NT} & 入力. 全波数の切断波数.\\
\texttt{D} & 出力. \texttt{SYCLAP}で用いられる配列.\\
\texttt{ICOM} & 入力. 計算に用いる MPIコミュニケーター.
\end{tabular}

\item 備考
    
\end{enumerate}

%---------------------------------------------------------------------

\subsection{SYCLAP}

\begin{enumerate}

\item 機能 
スペクトルデータにラプラシアンを作用させる, またはその逆演算を行う.

\item 定義

球面調和関数展開(ただし, ここでは $n$方向の切断波数を$N_T$とする)
\begin{equation}
g(\lambda,\varphi)=\sum^M_{m=-M}\sum^{N_T}_{n=|m|}
s^m_nP^m_n(\sin\varphi)e^{im\lambda}.
\end{equation}
に対して, 水平Laplacian
\begin{equation}
\nabla^2\equiv
\frac{\partial^2}{\cos^2\varphi\partial\lambda^2}
+\frac{\partial}{\cos\varphi\partial\varphi}\left(\cos\varphi\frac{\partial}{\partial\varphi}\right)
\end{equation}
を作用させると, 球面調和関数の性質から, 
\begin{equation}
\nabla^2 g(\lambda,\varphi)
=\sum^M_{m=-M}\sum^{N_T}_{n=|m|}-n(n+1)a^m_nP^m_n(\sin\varphi)e^{im\lambda}.
\end{equation}
となる. そこで,
\begin{equation}
(s_l)^m_n\equiv -n(n+1)s^m_n
\end{equation}
を導入すると, 
\begin{equation}
\nabla^2 g(\lambda,\varphi)
=\sum^M_{m=-M}\sum^{N_T}_{n=|m|}(s_l)^m_nP^m_n(\sin\varphi)e^{im\lambda}.
\end{equation}
と表せる. 
また, 逆に
\begin{equation}
\nabla^2 g(\lambda,\varphi)
=\sum^M_{m=-M}\sum^{N_T}_{n=|m|}s^m_nP^m_n(\sin\varphi)e^{im\lambda}.
\end{equation}
であるとき, 
\begin{equation}
(s_l)^m_n\equiv -\frac1{n(n+1)}s^m_n
\end{equation}
を導入すると, 定数分の任意性を除き,
\begin{equation}
g(\lambda,\varphi)
=\sum^M_{m=-M}\sum^{N_T}_{n=\max(1,|m|)}
(s_l)^m_nP^m_n(\sin\varphi)e^{im\lambda}
\end{equation}
と表せる. 

本サブルーチンは,
$s^m_n$から$(s_l)^m_n = -n(n+1)s^m_n$の計算, 
またはその逆演算: $s^m_n$から$(s_l)^m_n = -s^m_n/(n(n+1))$, を
行うものである. 

\item 呼び出し方法 
    
\texttt{SYCLAP(MM,NT,S,SL,D,IFLAG,ICOM)}
  
\item パラメーターの説明

\begin{verbatim}
INTEGER(8) :: MM, NT, IFLAG, ICOM
REAL(8) :: S((MM/NP+1)*(2*(NT+1)-MM/NP*NP)), SL((MM/NP+1)*(2*(NT+1)-MM/NP*NP))
REAL(8) :: D((MM/NP+1)*(2*(NT+1)-MM/NP*NP)*2)
\end{verbatim}

\begin{tabular}{ll}
\texttt{MM} & 入力. 帯状波数の切断波数.\\  
\texttt{NT} & 入力. 全波数の切断波数.\\
\texttt{S} & 入力. $s^m_n$が格納されている配列.\\
\texttt{SL} & 出力. $(s_l)^m_n$が格納される配列.\\
\texttt{D} & \texttt{SYINID}で初期化された配列.\\
\texttt{IFLAG} & 入力. \texttt{IFLAG=1}のときラプラシアンの演算を,
\texttt{IFLAG=2}のとき逆演算を行う.\\
\texttt{ICOM} & 入力. 計算に用いる MPIコミュニケーター.
\end{tabular}

\item 備考

(a) \texttt{IFLAG=2}の場合, $n=0$となる $(s_l)^0_0$ については
$s^0_0$がそのまま代入される.

\end{enumerate}

%---------------------------------------------------------------------

\subsection{SYCRPK}

\begin{enumerate}

\item 機能

全波数の切断波数($N_1$)のスペクトルデータを
全波数の切断波数($N_2$)のスペクトルデータに詰め替える.

\item 定義

\texttt{SYCS2Y}の出力結果と\texttt{SYCS2X}の出力結果を合成したいことは
しばしば生じるが, 両者は$n$方向の切断波数$N$が異なるスペクトルデータ
であるため, そのまま足すことができない . 本サブルーチンは, そのような
用途に応えるために,
全波数の切断波数($N_1$)のスペクトルデータ($(s_1)^m_n$)から,から
全波数の切断波数($N_2$)のスペクトルデータ($(s_2)^m_n$)への詰め替えを行うものである.
大きな領域への詰め替え($N_1<N_2$)の場合は,
($(s_2)^m_n$)の$n>N_1$の部分には$0$が代入される.
また, 小さな領域への詰め替え($N_1>N_2$)の場合は,
($(s_1)^m_n$)の$n>N_2$の部分の情報は捨てられる.
なお, ($N_1=N_2$)の場合は, 単なるコピーが行われる.

\item 呼び出し方法 
    
\texttt{SYCRPK(MM,NT1,NT2,S1,S2,ICOM)}
  
\item パラメーターの説明

\begin{verbatim}
INTEGER(8) :: MM, NT1, NT2, ICOM
REAL(8) :: S1((MM/NP+1)*(2*(NT1+1)-MM/NP*NP)), S2((MM/NP+1)*(2*(NT2+1)-MM/NP*NP))
\end{verbatim}
  
\begin{tabular}{ll}
\texttt{MM} & 入力. 帯状波数の切断波数.\\  
\texttt{NT1} & 入力. 入力の全波数の切断波数($N_1$).\\
\texttt{NT2} & 入力. 出力の全波数の切断波数($N_2$).\\  
\texttt{S1} & 入力. $(s_1)^m_n$が格納されている配列.\\
\texttt{S2} & 出力. $(s_2)^m_n$が格納される配列.\\
\texttt{ICOM} & 入力. 計算に用いる MPIコミュニケーター.
\end{tabular}

\item 備考

\end{enumerate}  
%---------------------------------------------------------------------

\subsection{SYKINI}

\begin{enumerate}

\item 機能
\texttt{SYPACK}の多層用初期化ルーチン.

\item 定義

多層モデル等において, 層の数を \texttt{KM}とするとき, 
各層(層方向の番号を\texttt{K}とする)に対して
スペクトルデータとグリッドデータの配列をそれぞれ 
{\texttt S(:, K)}, {\texttt G(:, K)}のように用意して
それら相互の変換を行いたい場合を考える. 
このとき, 全体のコミュニケーターを層方向に分割し,
分割されたより小さなコミュニケーターで各層の変換を
計算した方が並列化の効率が良い.

本サブルーチンは, 層の数と層方向の分割数を与えて, 
全体のコミュニケーター\texttt{ICOML}から
分割された小さなコミュニケーター(のうちで現在の
プロセスが属するもの)\texttt{ICOMS}を返すもの
である. なお, このコミュニケーターで計算される
層番号の範囲の下限\texttt{K1}と上限\texttt{K2}も同時に返す.
  
\item 呼び出し方法 
    
\texttt{SYKINI(KM,NDV,K1,K2,ICOML,ICOMS)}
  
\item パラメーターの説明

\begin{verbatim}  
INTEGER(8) :: KM, NDV, K1, K2, ICOML, ICOMS
\end{verbatim}  
    
\begin{tabular}{ll}
\texttt{KM} & 入力. 層の数.\\
\texttt{NDV} & 入力. 層方向の分割数.\\
\texttt{K1} & 出力. 分割されたコミュニケーターが担当する層番号の下端.\\
\texttt{K2} & 出力. 分割されたコミュニケーターが担当する層番号の上端.\\
\texttt{ICOML} & 入力. 多層計算全体に用いる MPIコミュニケーター.\\
\texttt{ICOMS} & 出力. 分割された MPIコミュニケーター.
\end{tabular}

\item 備考

(a) 分割前のコミュニケーター\texttt{ICOML}に属するプロセス数(サイズ)
  を\texttt{NPL}, 層数を\texttt{KM}とするとき, 分割数\texttt{NDV}は
  \texttt{NDV}$\le \min($\texttt{NPL}, \texttt{KM}$)$ を満して
  いなければならない.

(b) \texttt{ICOML}は計算に用いる分割前の MPIコミュニケーターを指定する.
 MPI の Fortranでの実装では, 通常, MPIコミュニケーターは
 \texttt{INTEGER(4)}型となっているので, 適宜\texttt{INTEGER(8)}に
キャストして入力とすること.

(c) 多層のスペクトルデータとグリッドデータの間の変換においては,
ここで与えられた\texttt{ICOMS}を\texttt{SYTS2G}等での入力
\texttt{ICOM}として用いればよい. ただし, 層方向のループは自前で
用意すること. すなわち, 以下のような感じになる.

\begin{verbatim}  
  CALL SYKINI(KM,NDV,K1,K2,ICOML,ICOMS)
  CALL SYINI1(MM,NM,IM,IT,T,R,ICOMS)
  CALL SYINI2(MM,NM,JM,1_8,P,R,JC,ICOMS)
  DO K=1,K2-K1+1
    CALL SYTS2G(MM,NM,NN,IM,JM,JV,S(1,K),G(1,K),IT,T,P,R,JC,W,IPOW,ICOMS)
  END DO
\end{verbatim}  
ここに, スペクトルデータ\texttt{S}とグリッドデータ\texttt{G}は
\begin{verbatim}        
REAL(8) :: S((MM/NP+1)*(2*(NN+1)-MM/NP*NP), (KM-1)/NDV+1)
REAL(8) :: G(IM*((JM/JV-1)/NP+1)*JV, (KM-1)/NDV+1)
\end{verbatim}      
のように用意しておけばよい. この場合, \texttt{NP}は
分割後のコミュニケーター\texttt{ICOMS}のサイズ(プロセス数)
となることに注意すること. これは, \texttt{ICOMS}に対して
\texttt{MPI\_COMM\_SIZE}を用いることで取得できる.
\end{enumerate}

%---------------------------------------------------------------------

\subsection{SYKGXX}

\begin{enumerate}

\item 機能 

多層用のデータ集約  

\item 定義

分割されたコミュニケーターのプロセスに分散配置された層方向のデータを
  データをランク0のプロセスに集約する.

\item 呼び出し方法 

\texttt{SYKGXX(NDIM,KM,NDV,X,XALL,ICOML)}
  
\item パラメーターの説明

\begin{verbatim}        
INTEGER(8) :: NDIM, KM, NDV, ICOML
REAL(8) ::  X(NDIM,(KM-1)/NDV+1), XALL(NDIM,KM)
\end{verbatim}      

\begin{tabular}{lll}
\texttt{NDIM} & 入力. \texttt{X}と\texttt{XALL}の第一次元寸法.\\
\texttt{KM} & 入力. 層の数.\\
\texttt{NDV} & 入力. 層方向の分割数.\\
\texttt{X} & 入力. 分散配置されたデータが格納されている配列.\\
\texttt{XALL} & 出力. 集約されたデータが格納される配列.\\
\texttt{ICOML} & 入力. 多層計算全体に用いる MPIコミュニケーター.
\end{tabular}

\item 備考

(a) \texttt{KM}, \texttt{NDV}, \texttt{ICOML}の意味については
   \texttt{SYKINI}の項を参照.

(b) このサブルーチンはすべてのプロセスで CALLされる必要があるが,
\texttt{XALL}の領域は, コミュニケーター\texttt{ICOML}の
ランク0のプロセスでのみ確保されていれば良い.

(c) \texttt{NDIM}は, スペクトルデータの場合は
\texttt{(MM/NP+1)*(2*(NN+1)-MM/NP*NP)}, グリッドデータの場合は
\texttt{IM*((JM/JV-1)/NP+1)*JV}とすればよい.

\end{enumerate}

%---------------------------------------------------------------------

\subsection{SYKSXX}

\begin{enumerate}

\item 機能 

多層用のデータの分散

\item 定義

ランク0のプロセスに集約された層方向のデータを
分割されたコミュニケーターのプロセスに分散配置する.

\item 呼び出し方法 

\texttt{SYKSXX(NDIM,KM,NDV,XALL,X,ICOML)}
  
\item パラメーターの説明

\begin{verbatim}        
INTEGER(8) :: NDIM, KM, NDV, ICOML
REAL(8) ::  X(NDIM,(KM-1)/NDV+1), XALL(NDIM,KM)
\end{verbatim}      

\begin{tabular}{lll}
\texttt{NDIM} & 入力. \texttt{X}と\texttt{XALL}の第一次元寸法.\\
\texttt{KM} & 入力. 層の数.\\
\texttt{NDV} & 入力. 層方向の分割数.\\
\texttt{XALL} & 入力. 集約されたデータが格納されている配列.\\
\texttt{X} & 出力 分散配置されたデータが格納される配列.\\
\texttt{ICOML} & 入力. 多層計算全体に用いる MPIコミュニケーター.
\end{tabular}

\item 備考

(a) \texttt{KM}, \texttt{NDV}, \texttt{ICOML}の意味については
   \texttt{SYKINI}の項を参照.

(b) このサブルーチンはすべてのプロセスで CALLされる必要があるが,
\texttt{XALL}の領域は, コミュニケーター\texttt{ICOML}の
ランク0のプロセスでのみ確保されていれば良い.

(c) \texttt{NDIM}は, スペクトルデータの場合は
\texttt{(MM/NP+1)*(2*(NN+1)-MM/NP*NP)}, グリッドデータの場合は
\texttt{IM*((JM/JV-1)/NP+1)*JV}とすればよい.

\end{enumerate}

%---------------------------------------------------------------------

\subsection{SYQRJV}

\begin{enumerate}

\item 機能

\texttt{JV}の値の推奨値の問い合わせ  

\item 定義

グリッドデータの大きさの設定に必要となる\texttt{JV}の値の推奨値を返す.

\item 呼び出し方法 
    
\texttt{SYQRJV(JM,JV)}
  
\item パラメーターの説明

\begin{verbatim}
INTEGER(8) :: JM, JV
\end{verbatim}    

\begin{tabular}{lll}
\texttt{JM} & 入力. 南北格子点数.\\  
\texttt{JV} & 出力. 推奨される\texttt{JV}の値.
\end{tabular}

\item 備考

(a) 返される\texttt{JV}の値は, \texttt{JM}の値および
ISPACKのインストール時 SSEの設定に依存して計算効率の良い値が設定される.

\end{enumerate}

\end{document}

