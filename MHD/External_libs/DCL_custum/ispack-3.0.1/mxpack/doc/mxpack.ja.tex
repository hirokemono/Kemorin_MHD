%***********************************************************************
% ISPACK FORTRAN SUBROUTINE LIBRARY FOR SCIENTIFIC COMPUTING
% Copyright (C) 1998--2019 Keiichi Ishioka <ishioka@gfd-dennou.org>
%
% This library is free software; you can redistribute it and/or
% modify it under the terms of the GNU Lesser General Public
% License as published by the Free Software Foundation; either
% version 2.1 of the License, or (at your option) any later version.
%
% This library is distributed in the hope that it will be useful,
% but WITHOUT ANY WARRANTY; without even the implied warranty of
% MERCHANTABILITY or FITNESS FOR A PARTICULAR PURPOSE.  See the GNU
% Lesser General Public License for more details.
% 
% You should have received a copy of the GNU Lesser General Public
% License along with this library; if not, write to the Free Software
% Foundation, Inc., 51 Franklin Street, Fifth Floor, Boston, MA
% 02110-1301 USA.
%***********************************************************************
\documentclass[a4j]{jsarticle}

\title{MXPACK使用の手引}
\author{}
\date{}

\begin{document}

\maketitle

\section{概要}

これは補助的な雑多(miscllaneous)なサブルーチンを集めたパッケージである.

%%%%%%%%%%%%%%%%%%%%%%%%%%%%%%%%%%%%%%%%%%%%%%%%%%%%%%%%%%%%%%%%%%%%%%

\section{サブルーチンのリスト}

\vspace{1em}
\begin{tabular}{ll}
\texttt{MXTIME} & 現在のUNIX-timeを取得する.\\
\texttt{MXALLC} & 64バイトメモリ境界にアラインされたメモリ領域を確保する.\\
\texttt{MXFREE} & メモリ領域の開放.\\
\texttt{MXGCPU} & ISPACKがインストールされた際の SSEの設定の問い合わせ.\\
\texttt{MXSOMP} & OpenMP並列化におけるスレッド数の指定.\\
\texttt{MXGOMP} & OpenMP並列化におけるスレッド数の取得.
\end{tabular}

%%%%%%%%%%%%%%%%%%%%%%%%%%%%%%%%%%%%%%%%%%%%%%%%%%%%%%%%%%%%%%%%%%%%%%

\section{サブルーチンの説明}

\subsection{MXTIME}

\begin{enumerate}

\item 機能 

UNIX時間(1970年01月01日00時00分00秒(UTC)からの時間)を秒単位で
返す.

\item 定義

\item 呼び出し方法 
    
\texttt{MXTIME(SEC)}
  
\item パラメーターの説明

  \vspace{-2ex}

\begin{verbatim}
REAL(8) :: SEC
\end{verbatim}

  \vspace{-1ex}
  
\begin{tabular}{ll}
\texttt{SEC} & 出力. 現在のUNIX時間(秒単位)
\end{tabular}

\item 備考

\end{enumerate}

%----------------------------------------------------------------------

\subsection{MXALLC}

\begin{enumerate}

\item 機能 

64バイトメモリ境界にアラインされたメモリ領域を確保する.

\item 定義

Intel x86 CPU上で, SSE=avx,  SSE=fma, または SSE=avx512
の設定で ISPACKを
make した場合, FXPACK, LXPACK, および SXPACK を利用するため
にはいくつかの配列の先頭が
メモリ上で 32バイト境界に割り当てられていなければならない
(SSE=avx512 の場合は 64バイト境界)が,
これをコンパイラが必ずしも自動ではやってくれるとは限らない
(Intel ifort では, -align array64byte オプションを付ければ
可能だが, gfortranではそのようなオプションが無い).
本サブルーチンは, Fortran90環境においてそのようなメモリ領域
の確保を可能にするためのものである.

\item 呼び出し方法 
    
\texttt{MXALLC(P,N)}
  
\item パラメーターの説明

\begin{verbatim}
TYPE(C_PTR) :: P
INTEGER(8) :: N
\end{verbatim}
      
\begin{tabular}{ll}
\texttt{P} & 出力. 確保されたメモリ領域の先頭のポインタが格納された
            \texttt{C\_PTR}型の構造体.\\
\texttt{N} & 入力. 確保すべき倍精度実数配列の大きさ.
      すなわち, \texttt{N*8} バイトのメモリ領域が確保される.
\end{tabular}

\item 備考

Fortran90のプログラムにおいて, 以下のように \texttt{ISO\_C\_BINDING}モジュー
ルを用いると, \texttt{MXALLC}を使って以下のように64バイト境界にアライン
された倍精度配列が利用できる. なお, 下記のプログラム例のように,
確保された領域は, 必要が無くなった段階で \texttt{MXFREE}で開放すること.

\begin{verbatim}
      USE ISO_C_BINDING
      IMPLICIT NONE
      INTEGER(8),PARAMETER :: M=20,N=10
      INTEGER(8) :: I,J
      REAL(8),DIMENSION(:,:),POINTER:: A
      TYPE(C_PTR) :: P

      CALL MXALLC(P,M*N) ! 領域の確保
      CALL C_F_POINTER(P,A,[M,N]) ! ポインタを配列に関連づけ
      ! これ以降, 倍精度実数配列 A(M,N)が使える.

      DO J=1,N
         DO I=1,M
            A(I,J)=1000*I+J
            PRINT *,A(I,J)
         END DO
      END DO

      CALL MXFREE(P) ! MXALLC で確保された領域の開放.
        
      END
\end{verbatim}

\end{enumerate}

%----------------------------------------------------------------------
\subsection{MXFREE}

\begin{enumerate}

\item 機能 

メモリ領域の開放.

\item 定義

\texttt{MXALLC}で確保されたメモリ領域を開放する.

\item 呼び出し方法 
    
\texttt{MXFREE(P)}
  
\item パラメーターの説明

\begin{verbatim}
TYPE(C_PTR) :: P
\end{verbatim}
  
\begin{tabular}{ll}
\texttt{P} & 入力. \texttt{MXALLC}でメモリを
確保した際に返されたポインタを格納した \texttt{C\_PTR}型構造体.
\end{tabular}

\item 備考

使用例は \texttt{MLALLC}の項を参照.

\end{enumerate}

%---------------------------------------------------------------------

\subsection{MXGCPU}

\begin{enumerate}

\item 機能 

ISPACKがインストールされた際の SSEの設定の問い合わせ

\item 定義

\item 呼び出し方法 

\texttt{MXGCPU(ICPU)}

\item パラメーターの説明

\begin{verbatim}
INTEGER(8) :: ICPU
\end{verbatim}

\begin{tabular}{ll}
\texttt{ICPU} & 出力. SSEの設定に対応する値(備考参照).
\end{tabular}

\item 備考

  (a) SSE=fort のとき \texttt{ICPU} = 0 が, 
  SSE=avx のとき \texttt{ICPU} = 10 が,
  SSE=fma のとき \texttt{ICPU} = 20 が,
  SSE=avx512 のとき \texttt{ICPU} = 30 が, 
  それぞれ返される.
  
\end{enumerate}


%---------------------------------------------------------------------

\subsection{MXSOMP}

\begin{enumerate}

\item 機能 

OpenMP並列化におけるスレッド数の指定.

\item 定義

ISPACK中で OpenMP並列化された部分を実行する際に使う最大のスレッド数を指定
する.

\item 呼び出し方法 

\texttt{MXSOMP(NTH)}

\item パラメーターの説明

\begin{verbatim}
INTEGER(8) :: NTH
\end{verbatim}

\begin{tabular}{ll}
\texttt{NTH} & 入力. 最大スレッド数.
\end{tabular}

\item 備考

(a) 環境変数 \texttt{OMP\_NUM\_THREADS} よりも \texttt{NTH}が
   大きい場合は, \texttt{OMP\_NUM\_THREADS}の値の方が設定される.
  
\end{enumerate}

%---------------------------------------------------------------------

\subsection{MXSOMP}

\begin{enumerate}

\item 機能 

OpenMP並列化におけるスレッド数の取得.

\item 定義

\texttt{MXSOMP}で設定されたスレッド数の取得する.
  
\item 呼び出し方法 

\texttt{MXGOMP(NTH)}

\item パラメーターの説明

\begin{verbatim}
INTEGER(8) :: NTH
\end{verbatim}

\begin{tabular}{ll}
\texttt{NTH} & 出力. 最大スレッド数.
\end{tabular}

\item 備考

(a) 事前に\texttt{MXSOMP}で設定がなされていない場合は
  環境変数 \texttt{OMP\_NUM\_THREADS}の値が \texttt{NTH}として返される.

\end{enumerate}


\end{document}
