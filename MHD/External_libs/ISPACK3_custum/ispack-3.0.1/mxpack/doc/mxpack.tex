%***********************************************************************
% ISPACK FORTRAN SUBROUTINE LIBRARY FOR SCIENTIFIC COMPUTING
% Copyright (C) 1998--2019 Keiichi Ishioka <ishioka@gfd-dennou.org>
%
% This library is free software; you can redistribute it and/or
% modify it under the terms of the GNU Lesser General Public
% License as published by the Free Software Foundation; either
% version 2.1 of the License, or (at your option) any later version.
%
% This library is distributed in the hope that it will be useful,
% but WITHOUT ANY WARRANTY; without even the implied warranty of
% MERCHANTABILITY or FITNESS FOR A PARTICULAR PURPOSE.  See the GNU
% Lesser General Public License for more details.
% 
% You should have received a copy of the GNU Lesser General Public
% License along with this library; if not, write to the Free Software
% Foundation, Inc., 51 Franklin Street, Fifth Floor, Boston, MA
% 02110-1301 USA.
%***********************************************************************
\documentclass[a4paper]{scrartcl}

\title{Manual of MXPACK}
\author{}
\date{}

\begin{document}

\maketitle

\section{Outline}

This is a package of miscllaneous subroutines.

%%%%%%%%%%%%%%%%%%%%%%%%%%%%%%%%%%%%%%%%%%%%%%%%%%%%%%%%%%%%%%%%%%%%%%

\section{List of subroutines}

\vspace{1em}
\begin{tabular}{ll}
\texttt{MXTIME} & get the current UNIX-time\\
\texttt{MXALLC} & allocate a memory space aligned with 64byte boundary\\
\texttt{MXFREE} & free a memory space\\
\texttt{MXGCPU} & Inquire the setting of SSE on installation of ISPACK\\
\texttt{MXSOMP} & Set the number of threads for OpenMP parallelization\\
\texttt{MXGOMP} & Get the number of threads for OpenMP parallelization
\end{tabular}

%%%%%%%%%%%%%%%%%%%%%%%%%%%%%%%%%%%%%%%%%%%%%%%%%%%%%%%%%%%%%%%%%%%%%%

\section{Usage of each subroutine}

\subsection{MXTIME}

\begin{enumerate}

\item Purpose

Show the UNIX-time, the number of seconds that have elapsed since
00:00:00 UTC, 1 January 1970.

\item Definition

\item Synopsis
    
\texttt{MXTIME(SEC)}
  
\item Parameters

\begin{verbatim}
REAL(8) :: SEC
\end{verbatim}
  
\begin{tabular}{ll}
\texttt{SEC} & Output. the current UNIX-time (in seconds).
\end{tabular}

\item Remarks

\end{enumerate}

%----------------------------------------------------------------------

\subsection{MXALLC}

\begin{enumerate}

\item Purpose

Allocate a 64-byte aligned memory.

\item Definition

If you have made ISPACK with setting SSE=avx,  SSE=fma, or SSE=avx512
on Intel x86 CPUs, some arrays for FVPACK, LVPACK, and
SVPACK must be 32-byte aligned, or must be 64-byte aligned
with setting SSE=avx512.
However, not all compilers
do the alignment automatically (Intel ifort can do it 
with the option ``-align array64byte'', but gfortran does
not have such an option).
This subroutine enables such a memory allocation on a 
general Fortran90 environment.

\item Synopsis
    
\texttt{MXALLC(P,N)}
  
\item Parameters

\begin{verbatim}
TYPE(C_PTR) :: P
INTEGER(8) :: N
\end{verbatim}
    
\begin{tabular}{ll}
\texttt{P} & Output. 
A \texttt{C\_PTR}-type structure which contains the pointer \\
& 
to the beginning address of the allocated memory.\\
\texttt{N} & Input. The size of the double-precision array
to be allocated \\
&    (namely, \texttt{N*8}-bytes of memory is allocated).
\end{tabular}

\item Remarks

In an Fortran90 program, you can use 64-byte aligned 
double-precision arrays by using \texttt{MXALLC} with 
the \texttt{ISO\_C\_BINDING} module as the following example.
You must free the allocated memory by calling \texttt{MXFREE}
before ending the program.

\begin{verbatim}
      USE ISO_C_BINDING
      IMPLICIT NONE
      INTEGER,PARAMETER :: M=20,N=10
      INTEGER :: I,J
      REAL(8),DIMENSION(:,:),POINTER:: A
      TYPE(C_PTR) :: P

      CALL MXALLC(P,M*N) ! memory allocation 
      CALL C_F_POINTER(P,A,[M,N]) ! associate a pointer with an array
      ! hereinafter, you can use a double-precision array A(M,N).

      DO J=1,N
         DO I=1,M
            A(I,J)=1000*I+J
            PRINT *,A(I,J)
         END DO
      END DO

      CALL MXFREE(P) ! free the memory that is allocated by MXALLC
        
      END
\end{verbatim}

\end{enumerate}

%----------------------------------------------------------------------
\subsection{MXFREE}

\begin{enumerate}

\item Purpose 

Free a memory space.

\item Definition

Free the memory allocated by \texttt{MXALLC}.

\item Synopsis 
    
\texttt{MXFREE(P)}
  
\item Parameters

\begin{verbatim}
TYPE(C_PTR) :: P
\end{verbatim}
  
    
\begin{tabular}{ll}
\texttt{P} & Input. 
The \texttt{C\_PTR}-type structure that contains  the pointer\\
 & when a memmory is allocated by \texttt{MXALLC}.
\end{tabular}

\item Remark

See also \texttt{MXALLC} for the usage example.

\end{enumerate}

%---------------------------------------------------------------------

\subsection{MXGCPU}

\begin{enumerate}

\item Purpose

Inquire the setting of SSE on installation of ISPACK  

\item Definition

\item Synopsis

\texttt{MXGCPU(ICPU)}  
  
\item Parameters

\begin{verbatim}  
INTEGER(8) :: ICPU
\end{verbatim}  

\begin{tabular}{ll}
\texttt{ICPU} & Output. A value corresponding to the SSE setting (see Remark).
\end{tabular}

\item Remark

  (a) If SSE=fort, then \texttt{ICPU} = 0; 
  if SSE=avx, then \texttt{ICPU} = 10;
  if SSE=fma, then \texttt{ICPU} = 20;
  if SSE=avx512, then\texttt{ICPU} = 30.

\end{enumerate}

%---------------------------------------------------------------------

\subsection{MXSOMP}

\begin{enumerate}

\item Purpose  

Set the number of threads for OpenMP parallelization.

\item Definition
  
Set the maximum number of threads to execute parallelized parts by 
  OpenMP in ISPACK.

\item Synopsis

\texttt{MXSOMP(NTH)}

\item Parameters

\begin{verbatim}
INTEGER(8) :: NTH
\end{verbatim}

\begin{tabular}{ll}
\texttt{NTH} & Input. Maximum number of threads.
\end{tabular}

\item Remarks

(a) If \texttt{NTH} is larger than \texttt{OMP\_NUM\_THREADS},
   the maximum number of threads is set to \texttt{OMP\_NUM\_THREADS}.
  
\end{enumerate}

%---------------------------------------------------------------------

\subsection{MXGOMP}

\begin{enumerate}

\item Purpose  

Get the number of threads for OpenMP parallelization.

\item Definition

Get the number of threads set by \texttt{MXSOMP}.

\item Synopsis

\texttt{MXGOMP(NTH)}

\item Parameters

\begin{verbatim}
INTEGER(8) :: NTH
\end{verbatim}

\begin{tabular}{ll}
\texttt{NTH} & Output. Maximum number of threads.
\end{tabular}

\item Remarks
  
  (a) If \texttt{MXSOMP} is not called beforehand,
  \texttt{NTH} is set to \texttt{OMP\_NUM\_THREADS}.

\end{enumerate}

\end{document}
